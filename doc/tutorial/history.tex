% file: history.tex
% 	CORE Library
% 	$Id: history.tex,v 1.1.1.1 2007/04/09 23:35:54 exact Exp $ 

\documentclass[12pt]{article}

\usepackage{amssymb}
% Macros for CORE Documents 

%%%%%%%%%%%%%%%%%%%%%%%%%%%%%%%%%%%%%%%%%%%%
% UNDATE THESE MACROS FOR EACH RELEASE!
%%%%%%%%%%%%%%%%%%%%%%%%%%%%%%%%%%%%%%%%%%%%
% Current Core Library Version
\newcommand{\versionNo}{1.7}
% Current Release Date
\newcommand{\releaseDate}{August 15, 2004}

%%%%%%%%%%%%%%%%%%%%%%%%%%%%%%%%%%%%%%%%%%%%
\newcommand{\ignore}[1]{}
\newcommand{\dt}[1]{{\em #1}}
\newcommand{\RR}{{\mathbb R}}

\newcommand{\corelib}{Core Library}
\newcommand{\core}{\mbox{\tt CORE}}
\newcommand{\corex}{\mbox{\tt COREX}}

\newcommand{\lidia}{\mbox{\tt LiDIA}}
\newcommand{\cgal}{\mbox{\tt CGAL}}
\newcommand{\cln}{\mbox{\tt CLN}}

\newcommand{\rexpr}{\mbox{\tt Real/Expr}}
\newcommand{\real}{\mbox{\tt Real}}
\newcommand{\realrep}{\mbox{\tt RealRep}}
\newcommand{\expr}{\mbox{\tt Expr}}
\newcommand{\exprep}{\mbox{\tt ExprRep}}

\newcommand{\rep}{\mbox{\tt rep}}
\renewcommand{\int}{\mbox{\tt int}}
\newcommand{\lng}{\mbox{\tt long}}
\newcommand{\ulong}{\mbox{\tt unsigned long}}
\newcommand{\extlong}{\mbox{\tt extLong}}
\newcommand{\float}{\mbox{\tt float}}
\newcommand{\double}{\mbox{\tt double}}
\newcommand{\str}{\mbox{\tt string}}
\newcommand{\Int}{\mbox{\tt BigInt}}
\newcommand{\Rat}{\mbox{\tt BigRat}}
\newcommand{\BF}{\mbox{\tt BigFloat}}
\newcommand{\Poly}{\mbox{\tt Polynomial}}
\newcommand{\Sturm}{\mbox{\tt Sturm}}
\newcommand{\BiPoly}{\mbox{\tt BiPoly}}
\newcommand{\Curve}{\mbox{\tt Curve}}


% DO WE REALLY WANT "Long" below?
\newcommand{\coreInfty}{\mbox{\tt CORE\_INFTY}}
\newcommand{\posInfty}{\mbox{\tt extLong::CORE\_posInfty}}
\newcommand{\negInfty}{\mbox{\tt extLong::CORE\_negInfty}}
\newcommand{\NaN}{\mbox{\tt extLong::CORE\_NaN}}
\newcommand{\defrel}{\mbox{\tt defRelPrec}}
\newcommand{\defabs}{\mbox{\tt defAbsPrec}}
\newcommand{\defprt}{\mbox{\tt defPrtDgt}}
\newcommand{\definput}{\mbox{\tt defInputDigits}}
\newcommand{\defBFinput}{\mbox{\tt defBigFloatInputDigits}}
\newcommand{\defOutputDig}{\mbox{\tt defOutputDigits}}

\newcommand{\cpp}{\mbox{\tt C++}}
\newcommand{\candcpp}{\mbox{\tt C/C++}}

\newcommand{\gnu}{\mbox{\tt GNU}}
\newcommand{\gmp}{\mbox{\tt GMP}}
\newcommand{\gpp}{\mbox{\tt g++}}
\newcommand{\gppandgcc}{\mbox{\tt g++/gcc}}

\newcommand{\installpath}{\mbox{\tt \$\{INSTALL\_PATH\}}}
% \newcommand{\corepath}{\mbox{\tt \$\{INSTALL\_PATH\}/core\_vX.Y/}}
\newcommand{\corepath}{\mbox{\tt \$\{CORE\_PATH\}}}
\newcommand{\coredistfile}{\mbox{\tt core\_vX.Y.Z.tgz}}
\newcommand{\docdir}{\mbox{\tt \$\{CORE\_PATH\}/doc}}
\newcommand{\extdir}{\mbox{\tt \$\{CORE\_PATH\}/ext}}
\newcommand{\examplesdir}{\mbox{\tt \$\{CORE\_PATH\}/progs}}
\newcommand{\progsdir}{\mbox{\tt \$\{CORE\_PATH\}/progs}}
\newcommand{\includedir}{\mbox{\tt \$\{CORE\_PATH\}/inc}}
\newcommand{\includegmpdir}{\mbox{\tt \$\{CORE\_PATH\}/gmp/include}}
\newcommand{\libdir}{\mbox{\tt \$\{CORE\_PATH\}/lib}}
\newcommand{\gmplibdir}{\mbox{\tt \$\{CORE\_PATH\}/gmp/lib}}
\newcommand{\win}{\mbox{\tt \$\{CORE\_PATH\}/win32}}
% no longer used:
	% \newcommand{\lidiadir}{\mbox{\tt \$\{CORE\_PATH\}/lidia}}
\newcommand{\srcdir}{\mbox{\tt \$\{CORE\_PATH\}/src}}
\newcommand{\tempdir}{\mbox{\tt \$\{CORE\_PATH\}/tmp}}

\newcommand{\gmplink}{\mbox{\tt \$\{CORE\_PATH\}/gmp}}

\newcommand{\val}{\mbox{\rm Val}}
\newcommand{\err}{\mbox{\rm Err}}
\newcommand{\preci}{\mbox{\rm Prec}}
\newcommand{\appr}{\mbox{\rm Approx}}

\newenvironment{prog}{\begin{tabbing}
xx\=xxxx\=xxxx\=xxxx\=xxxx\=xxxx\=xxxx\=xxxx\=\kill \\}
{\end{tabbing}}

\newenvironment{progb}[1]{
\vspace{+\abovedisplayskip}
\fbox{
\begin{minipage}{.9\textwidth}
\vspace{-\abovedisplayskip}
\begin{prog}#1\end{prog}
\vspace{-\abovedisplayskip}
\end{minipage}
}
\vspace{+\abovedisplayskip}
}

\newcommand{\draftdimensions}{
        \setlength{\oddsidemargin}{-1cm}
        \setlength{\evensidemargin}{-1cm}
% \setlength{\textwidth}{1.2\textwidth}
        \setlength{\textwidth}{7in}
% \setlength{\textheight}{1.2\textheight}
        \setlength{\textheight}{9in}
        \setlength{\topmargin}{-5mm}
        \setlength{\parindent}{0mm}
        \setlength{\parskip}{5mm}
}



\draftdimensions


\title{What Was New In Core Library}


\begin{document}
\maketitle

\section*{}
This is an accumulated list of the
``What is New'' information, released with
each version of the Core Library.

\section{Version 1.1, Dec 1998}
The initial implementation
by Karamcheti, Li, Pechtchanski and Yap \cite{klpy:core:98}
was based on the \rexpr\ package,
designed by Dub\'e and Yap (circa 1993)
and rewritten by Ouchi \cite{ouchi:thesis}.  
Details about the underlying 
algorithms (especially in \BF) and their error analysis
may be found in the Ouchi's thesis \cite{ouchi:thesis}.

\section{Version 1.2, Sep 1999}
Incorporates the BFMS root bound and other 
techniques to give significant speedup to the system. 

   \begin{itemize}
   \item{New root bound implemented.}
   This is based on the paper of
   Burnikel et al \cite{mehlhorn-etal:sep-bd:00}. 
   
   \item{Dynamic error bound checking.}
   The system utilizes the error bound of the computed 
   approximate value to decide whether or not
   to re-evaluate a node in an expression.  This is faster than
   just looking at the precision bound at that node because
   the precision bound is only a lower bound on the computed
   precision.
   
   \item{Progressive and non-progressive evaluation.} 
   
   \item{Expression constructor from string.} 
   
   \item{Exception handling.}  We introduce a simple facility
   for \cpp\ exception handling. See the file
   \includedir/{\tt CoreExceptions.h}.

   \end{itemize}

\section{Version 1.3, Sep 2000}
Two improvements (new root bounds
and faster big number packages based on \lidia/\cln)
gave significant general speedup.  
More examples, including a hypergeometric function package
and a randomized geometric theorem prover.

	\begin{itemize}
	\item
	Implemented new root bounds as described in
	\cite{li-yap:constructiveBound:00}.  This 
	gave significant speedups for the important case
	of expressions with division nodes.
	\item
	Printout of numbers is improved so that all but the last digit
	of any printout is correct.
	The last digit can be off by $\pm 1$.
	\item
	Input of numbers (string constructors and stream input) is
	improved to allow both exact rational inputs as well as
	approximations.  Uses the global variable {\tt defInputDgt}.
	\item
	 A much faster big number kernel (taken from
	\lidia/\cln) is incorporated.  We can more easily
	incorporate other big number kernels (such as \gnu's \gmp),
	especially those supported by \lidia.
	Unfortunately, we have to drop support for those compilers not
	compatible with \lidia.
	\item
	Application program:
	Contribution of a hypergeometric package
        from Maria Elefteriou and Jose Moreira from IBM Watson Center.
        You can compute any of the elementary functions (log, exp,
        sine, etc) to any desired absolute precision.  This
	can be used to compute constants such as $\pi$ or $e$ to
	any desired precision.
	\item
	Application program:
	improved geometric theorem prover \cite{tyl:zero-test:00}
	based on randomized testing of radical expressions.
	\item
	Sample timing: it took 55.0 seconds under \core\ 1.3
	to run all the test programs in the distribution.
	Under \core\ 1.2.2, the time is 1040.38 seconds.
	This test was performed on a Sun UltraSPARC-IIi (440 MHz, 512 MB),
	and corresponds to the ``make test''
	command in the {\tt \$(CORE\_PATH)} directory.
	\end{itemize}

\section{Version 1.4, Sep 2001}
Our big integer and big rational packages
are now based on \gmp, away from \lidia/\cln.
Introduced a floating point filter facility,
incremental square root computation,
improved precision-sensitive evaluation algorithms.

	\begin{itemize}
        \item
	File I/O for big numbers.  In particular the storage of Hex
                representation in a files allows large constants to be
                input or output in linear time.
	\item
        Incremental square-roots now implemented.  There can be
                significant speedups.
 	\item
	BFS filter option
        \item
	Introduced the "escape precision" mechanism, and a recommended
        	{\tt CORE\_init()} method.
	\item
        Improved conversion from BigFloat to machine double: 
                overflows, etc, are indicated.
        \item
	Improved geometry extension library                
	\item
	Improved numerical I/O.
	I/O of big rational number format,
        BigInt constructor from string, support for scientific as well
	as positional notation.
	\item
        Bug fixes: Steve Robbins discovered that
		\texttt{defAbsPrec} is not always properly initialized.
                The solution is to make \texttt{CORE\_posInfty},
                \texttt{CORE\_negInfty} and \texttt{CORE\_NaNLong}
		global variables, not static member variables.
	\end{itemize}

\section{Version 1.5, Aug 2002}
Improvements in speed from better root bounds (k-ary bounds),
        CGAL compatibility changes, file I/O for large mathematical constants
        (BigInt, BigFloat, BigRat), improved hypergeometric package.

    \begin{itemize}
        \item
	Namespace Issues:
          Introduced namespace \texttt{CORE}.
       \item
	Compatibility with \cgal\ Library:
          Introduced \texttt{inc/CGAL\_Expr.h} for easy adaptation
	  of \cgal\ programs to use \corelib.  Sample programs are found
	  under in the cgal directory under \examplesdir.
       \item
         Better root bounds (including the new BFMSS bound 
	 and a new 2-ary version of BFMSS).  Various minor improvements.
	 Overall speedup can be greatly improved especially when
	 rootbound limits are reached.
          Some timings for various test programs found
          under \examplesdir (just based on the new BFMSS bound)
	  \begin{verbatim}
           (1) progs/pi, executing "./pi 10000 0 0"
                       Core 1.5: 0.79 sec
                       Core 1.4: 2.37 sec
                       Core 1.3: 2.93 sec
           (2) progs/compare, running "./compare 100 n" (for n=2 and n=5)
                       Core 1.5:       n=2, 0.01 sec;     n=5, 0.03 sec
                       Core 1.4:       n=2, 22.58 sec;    n=5, 56.61 sec
                       Core 1.3:       n=2, 36.12 sec;    n=5, 90.59 sec
           (3) progs/delaunay, executing "make test"
                       Core 1.5: 0.43 sec
                       Core 1.4: 0.43 sec
                       Core 1.3: 0.49 sec
           (4) progs/nestedSqrt, executing "./nestedSqrt 8"
                       Core 1.5: 0.62 sec
                       Core 1.4: 0.82 sec
                       Core 1.3: 447.29 sec
	 \end{verbatim}
           These tests were done on a Solaris machine, SunBlade 1000
               (2x Ultrasparc III 750MHz, 8MB Cache, 2GB Memory)

	\item
           Improvements of the hypergeometric package
               (automatic error analysis, argument reduction,
               partial parameter preprocessing, improved $\log(x)$ series)
               In particular, the elementary functions
               found in \texttt{math.h} are reasonably supported:
	       	$\log()$, $\exp()$, $\sin()$, $\cos()$, $\arcsin()$,
                $\arccos()$, $\tan()$, $\arctan()$, $\textrm{erf}()$.
        \item
	   Some development of other Levels of Accuracy:
	   (a) Level 4 can now be explicitly invoked.
	   (b) Level 2 programs can now compile without error:
		to achieve this, we have to merge the API for Expr
		and Real classes.
        \item
		New Polynomial Class (univariate case) may be found in
		\examplesdir, under \texttt{poly}, with
               applications to Sturm Sequences and Newton Iteration.
        \item
		Reorganization of \real\ class:
               new abstract base class \realrep\ (previously no such class).
               \real\ class is no longer virtual (no longer points to itself
                       or have reference counter, but points to \realrep).
                Streamlined code for \texttt{Real.cpp} by introducing templates.
		The classes \texttt{AddRep} and \texttt{SubRep} are now
		merged by templates, to improve maintenability.
        \item
		Implemented the ability to read/write arbitrarily 
		large \BF\ and \Rat\ constants from files
		(Version 1.4 only supported file I/O for \Int\ constants)
        \item
		Improved algorithm for computing degree bound (degreeBound(),
               count(), clearFlag()), avoiding exponential worst case behavior
        \item
		Removed implementation dependent files:
               \texttt{gmp-impl.h}, \texttt{longlong.h}, \texttt{gmp-params.h}.
		Also removed obsolete files:
               \texttt{ExprUtil.h,cpp}, \texttt{Filer.cpp},
	       \texttt{CoreIOmanip.h}, \texttt{CoreException.h}.
        \item
		Updated Core Library Tutorial.  
        \item
		Bug fixes:
		(a) Computing of uMSB and lMSB was inaccurate
		(b) Root bound parameters for some constants
			was wrong (e.g., for Expr(1.0)).
		(c) BigInt::ceilLg(x) was wrong for $x<0$.
		(d) Thanks to Martin Held for exposing a bug
		in the installation script (script may compile
		gmp library using a different compiler than the one
		used to compile Core Library, with the result that 
		linking may fail).  
	\end{itemize}

\section{Version 1.6, Jun 2003}
The introduction of real algebraic numbers.
CORE is now issued under the the Q PUBLIC LICENSE (QPL),
concurrently with its being distributed with CGAL.
Incorporated Polynomial and Sturm classes into Core Library.

    \begin{itemize}
	\item Two new classes have been incorporated into CORE: 
		Polynomial class and Sturm class.  Both are templated classes.
		The coefficients of a polynomial
		have a parameterized number type NT.  Currently NT
		is restricted to \Int, \int, \lng.
	\item An associated Sturm class has methods for root isolation
		and Newton iteration.
        \item Real algebraic numbers is now available as leaves
		in \expr.  These can be specified as a polynomial
		together with a root root indicator.
	   	The special case of m-th root of n is implemented as 
		radical(n,m).
	\item Further improvement in k-ary BFMSS Root Bound (we now extract
		powers of 5 (as well as powers of 2).  Thus decimal inputs
		can be exploited.
	\item By setting a global \texttt{rationalReduceFlag = true},
		you can automatically
		reduce rational \expr\ to a single node.  This allows some
		previously intractable computations to succeed (see pentagon
		example under progs directory).  By default, this flag is false.
	\item New wrapper classes for \Int\ and \Rat\ to
		directly exploit \gmp.  Previously it was based on
		\lidia's wrapper classes (even though the
		underlying number classes were already based on \gmp).
		This gives about 11.8\% speedup.
	\item Much faster algorithm for refining isolating interval. 
		This comes from two sources. First,
		instead of only bisection, we exploit Newton iteration 
		when we can. Second, we
		Replace \expr\ intervals by \BF\ intervals.
	\item Introduced the \texttt{defInitialProgressivePrec} variable,
		to control the initial
		precision for progressive evaluation:
		originally, this initial value was 1.
		Here are the speed from "make time" for variouse values of 
		defInitialProgressivePrec (based on Core v.1.6 Test Programs):
\begin{verbatim}
		defInitialProgressivePrec 	User time (seconds)
				1		1:11.5
				32 		1:09.4
				64 		1:08.7
				128		1:08.3
				256		1:09.3
\end{verbatim}

	\item Implemented \texttt{toString} and \texttt{fromString} methods for 
		\Int, \Rat, \BF\, \real, \expr.
	\item In \BF, we now have methods for \texttt{makeExact()},
		\texttt{makeInexact()}
		and \texttt{div2()} for error-free division by 2.
		These are important for controlling I/O precision
		and Newton iteration.
	\item Implemented automatic checks for invalid \expr\ at 
		construction time.  E.g.,
		for division, check for divide-by-zero, and for sqrt,
		check for negative argument.
	\item Various bug fixes: in particular, fixed a
		bug arising from negative root bit-bound 
		(this arises from the new k-ary root bounds, since
		previous root bound parameters could never generate
		negative root bit-bounds)
	\item Miscellanenous changes:
	\begin{itemize}
		\item Core Numerical API Level is now called
		\texttt{CORE\_LEVEL}.
		\item Restructuring of include files:
		we now prefer to use \texttt{\#include "CORE/xxx.h"}
		  instead of \texttt{\#include "xxx.h"} to avoid name
		  conflict.   For backwards compatibility, you can
		  still use \texttt{\#include "CORE.h"}.
		\item Reorganization of header files (e.g.,
		\texttt{Expr.h} is split into \texttt{Expr.h}
		and \texttt{ExprRep.h}). 
		\item New functions: \texttt{Expr::operator \%(x,y)},
		\texttt{BigFloat::pow(x,n)}
		\item New functions: \texttt{Expr::doubleInterval()}
		returns an interval containing exact value
		\item New \texttt{escapePrecisionWarning} boolean flag
		(useful for testing escape precision)
		\item New \texttt{gcd(long, long)} function
	\end{itemize}
    \end{itemize}

\section{Version 1.7, Aug 2004}

Introduced algebraic curves and bivariate polynomials.
An interactive version of Core Library called "InCore" is available.
Beginning some graphic capability for display of curves.  Restructuring
of number classes (Expr, BigFloat, etc) to have common reference counting
and rep facilities.

   \begin{itemize}
   \item Introduced plane algebraic curves and bi-variate polynomials 
         The main capability is to plot curves and do basic
         intersection tests.  See \examplesdir\texttt{/curves}.
   \item Introduced "InCore", an interactive version of Core Library 
   	based on Python (this has to be downloaded separately)
   \item Enhancement of the univariate polynomial and real algebraic
        number facilities. New methods such as
	polynomial GCD, resultant, square free part, primitive part, etc.
	\begin{itemize}
	\item Polynomials now accept string inputs. 
	 E.g., \texttt{Polynomial<BigInt> p = "x\^{}3 - 2x\^{}2 + 17x - 4";}
        \item \texttt{Polynomial<NT>} can now work with all choices of
                \texttt{NT = BigInt, int, BigRat, Expr, BigFloat}
          But not all functionality are fully available for
          \texttt{NT=BigFloat} and \texttt{NT=Expr} (e.g., rootbounds).
        \item \texttt{Sturm<NT>} can now work with \texttt{NT=BigFloat}
	\item To support the above, various new methods are added to the
	  \Int, \Rat, \BF\ and \real\ classes (isDivisible, gcd, etc).
	\end{itemize}
   \item Compatibility with gmp 4.1.4, and gcc 3.3

   \item Introduced a common Reference Counting facility for all
         Core number types, encoded in the two templated classes:
                \texttt{RCRepImpl<class N>}
      to create Reps of the class N.  The basic functions provided by
      this class is reference counting, and gives us the "Reps" of
      each number type (e.g., \texttt{BigIntRep}
      is derived from \texttt{RCRepImpl}).  
      The other class is
                \texttt{RCImpl<class T>}
      which is the actual number type (e.g., BigInt is derived
      from \texttt{RCImpl<BigIntRep}>).  As a result former Rep Files
      such as RealRep.h can be removed, and the code size is reduced.
      Also, \Int\ and \Rat\ now have reference counting in
      their Rep classes (none before).
   \item Speedup from reorganization of Core Library number classes.
      We wrote wrappers around gmp's C function library.
      We also tested the possibility of using gmp's C++ classes (since
      gmp 4.1) which are template based.  CORE's interface are not fully
      compatible with gmp since since we have exact algebraic representation
      and precision bounds.
      We compared three versions of Core Library:
           \\
           (A) -- old code: version 1.6
	   \\
           (B) -- new code: some optimization
	   \\
           (C) -- new code: optimzation + using gmp C++ class
      \\Here are the running times on two machines:
      \begin{verbatim}
      Test              Pentium III (1G Memory)        Jinai (Solaris)
      =======================================================
      > A               64.12s                         1m:07s
      > B               56.52s                         52.6s
      > C               50.71s                         50.4s
      > Speedup         20%                            25%
      \end{verbatim}
    
      Although the use of gmp C++ classes (which use expression template
      to eliminate temporary variables) is faster, it is
      incompatiable with visual C++ (not sure about Sun CC).
      So we added as an option for compiling the Core Library:
      to turn on gmp C++ classes, uncomment macro CORE\_USE\_GMPXX
      in CoreImpl.h and recompile Core Library; there is no need to compile
      GMP C++ library since all necessory code are in Core Library already.

   \item Simple openGL display for curves: see
   	\extdir\texttt{/graphics/}

   \item Various improvements by replacing Expr with BigFloats
   	whenever possible: in Polynomial and Sturm operations,
	in progs/pi/brent.cpp, etc.

   \item The compiled Core Library is changed from
   	"libcore" to "libcore++", and "libcorex" to "libcorex++".  
	This mitigates any possible name conflicts as a Debian package 
	for Core Library (see below).

   \item Rewrote the memory pool class to avoid strict aliasing
   	problems under gcc's -O2 optimization.  Now, we can compile
	Core Library under -O2 optimization.

   \item Bug fixes
        \begin{itemize}
          \item \texttt{rootOf(P,i)} now works properly when P have
	     multiple roots
            (reason: we assumed the endpoints of isolating intervals
            have distinct signs)
	  \item Fixed bugs in Newton and Sturm methods
          \item \texttt{Expr::doubleValue()} is now correctly implemented.
	    E.g., suppose you compute
	        double s = sqrt(n);
		double ss = sqrt(Expr(n)).doubleValue();
            Then we guarantee that |s-ss| has relative error at
	    most \texttt{4*CORE\_EPS = 4.44089e-16}
	    (\texttt{CORE\_EPS} is machine epsilon). 
	    This factor of 4 is essentially the best possible (see \\
	    \examplesdir\texttt{/testIO/testSqrt.cpp}).
	  \item fixed \texttt{Expr::degreeBound()}.  Previously it always 
	    returned 1 at
	    leaves.  But with algebraic numbers, it must return
	    \texttt{d\_e()}.
	    This has dramatic improvement in speed. Ron Wein's program for
	    intersecting two ellipses used to take overnight, but now take
	    0.4 sec.
	  \item Fixed an output bug that has been around since Core 1.4.
	    BigFloat can print an output whose exponent is off by 1.
	        E.g., \texttt{sqrt(100) = 9.999e0}
		but \texttt{BigFloatRep::round(...)},
		an internal function, returns the string "1.0000e0"
		instead of "1.0000e1".  Thanks to Blazi for noticing this.
        \end{itemize}
   \item Miscellaneous:
        \begin{itemize}
	  \item Thanks to Joachim Reichel for creating a Debian package
	    for Core Library.
          \item Previously, the library version numbers have two numbers
            (e.g., "1" and "6" in Version 1.6).  Between official releases,
	    we call it Version 1.6x.  Thus "Version 1.6x" refers to any
            number of unofficial releases between 1.6 and 1.7.
            We now have a third number.  E.g., this official release
            is Wersion 1.7.0.
          \item \texttt{CORE::core\_error()} is enhanced to write its results
	    into a file \texttt{Core\_Diagnostics} but also writes errors to
	    std::cerr (as before).  
	  \item We introduce new versions of the \texttt{BigFloat::makeExact()}
	    , namely, \texttt{makeCeilExact()} and \texttt{makeFloorExact()}.
          \item Upgraded Core Library to be compatible with the latest gmp
            4.1.3 (released 4/28/2004).
	  \item Use the program \texttt{astyle} to beautify all our code,
	    so the files are more consistent and readable.  Tab are 
	    converted to spaces, using k\&r style, etc:
	    	\begin{equation}
	               \texttt{./astyle --style=kr -s2 filename}
	    	\end{equation}
	  \item More use of BigFloats to replace Expr, when possible.  E.g.
	    CauchyLowerBound() and CauchyUpperBound() rewritten using
	    BigFloat instead of Expr (30\% improvement here).
	  \item In CORE\_PATH, you can type "make options" 
	    to see all your currently selected CORE ENVIRONMENT VARIABLES.
	    If you type "make alloptions", this will also show
	    all possible alternatives that you could have chosen.
	  \item Simple timing facility (see src/Timer.h)
        \end{itemize}
   \item Acknowledgements:  Thanks for feedback and bug reports from
            Janos Blazi,  Ovidiu Daescu, Arno Eigenwillig, Andreas Fabri,
	    Efi Fogel, Michael Hemmer, Athanasios Kakargias, Norbert Mueller,
	    Joachim Reichel, Daniel Russell, Raphael Straub, Ron Wein.

  \end{itemize} 
 
 \bibliographystyle{abbrv}
 \bibliography{tutorial}

\end{document}
