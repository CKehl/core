% file: tutorial.tex
% 	CORE Library
% 	$Id: tutorial.tex,v 1.1.1.1 2007/04/09 23:35:54 exact Exp $ 
% Revision History:
%	-- Nov 12, 2004 by Chee, Zilin, Vikram and Pion for Version 1.7
%	-- Jun 20, 2003 by Chee, Zilin and Vikram for CORE Version 1.6
%	-- Jul 29, 2002 by Sylvain, Chee, Zilin for CORE Version 1.5
%	-- Sep  1, 2001 by Chee for CORE Version 1.4x
%	-- Aug 15, 2000 by Chee and Chen for CORE Version 1.3
%	-- Sep  9, 1999 by Chee and Chen for CORE Version 1.2
% 	-- Jan  9, 1999 by Chen for a new tutorial for CORE lib
% 	-- Jun 15, 1998 by Chee for CORE proposal


\documentclass[12pt]{article}


% \input mssymb
\usepackage{amssymb}
% Macros for CORE Documents 

%%%%%%%%%%%%%%%%%%%%%%%%%%%%%%%%%%%%%%%%%%%%
% UNDATE THESE MACROS FOR EACH RELEASE!
%%%%%%%%%%%%%%%%%%%%%%%%%%%%%%%%%%%%%%%%%%%%
% Current Core Library Version
\newcommand{\versionNo}{1.7}
% Current Release Date
\newcommand{\releaseDate}{August 15, 2004}

%%%%%%%%%%%%%%%%%%%%%%%%%%%%%%%%%%%%%%%%%%%%
\newcommand{\ignore}[1]{}
\newcommand{\dt}[1]{{\em #1}}
\newcommand{\RR}{{\mathbb R}}

\newcommand{\corelib}{Core Library}
\newcommand{\core}{\mbox{\tt CORE}}
\newcommand{\corex}{\mbox{\tt COREX}}

\newcommand{\lidia}{\mbox{\tt LiDIA}}
\newcommand{\cgal}{\mbox{\tt CGAL}}
\newcommand{\cln}{\mbox{\tt CLN}}

\newcommand{\rexpr}{\mbox{\tt Real/Expr}}
\newcommand{\real}{\mbox{\tt Real}}
\newcommand{\realrep}{\mbox{\tt RealRep}}
\newcommand{\expr}{\mbox{\tt Expr}}
\newcommand{\exprep}{\mbox{\tt ExprRep}}

\newcommand{\rep}{\mbox{\tt rep}}
\renewcommand{\int}{\mbox{\tt int}}
\newcommand{\lng}{\mbox{\tt long}}
\newcommand{\ulong}{\mbox{\tt unsigned long}}
\newcommand{\extlong}{\mbox{\tt extLong}}
\newcommand{\float}{\mbox{\tt float}}
\newcommand{\double}{\mbox{\tt double}}
\newcommand{\str}{\mbox{\tt string}}
\newcommand{\Int}{\mbox{\tt BigInt}}
\newcommand{\Rat}{\mbox{\tt BigRat}}
\newcommand{\BF}{\mbox{\tt BigFloat}}
\newcommand{\Poly}{\mbox{\tt Polynomial}}
\newcommand{\Sturm}{\mbox{\tt Sturm}}
\newcommand{\BiPoly}{\mbox{\tt BiPoly}}
\newcommand{\Curve}{\mbox{\tt Curve}}


% DO WE REALLY WANT "Long" below?
\newcommand{\coreInfty}{\mbox{\tt CORE\_INFTY}}
\newcommand{\posInfty}{\mbox{\tt extLong::CORE\_posInfty}}
\newcommand{\negInfty}{\mbox{\tt extLong::CORE\_negInfty}}
\newcommand{\NaN}{\mbox{\tt extLong::CORE\_NaN}}
\newcommand{\defrel}{\mbox{\tt defRelPrec}}
\newcommand{\defabs}{\mbox{\tt defAbsPrec}}
\newcommand{\defprt}{\mbox{\tt defPrtDgt}}
\newcommand{\definput}{\mbox{\tt defInputDigits}}
\newcommand{\defBFinput}{\mbox{\tt defBigFloatInputDigits}}
\newcommand{\defOutputDig}{\mbox{\tt defOutputDigits}}

\newcommand{\cpp}{\mbox{\tt C++}}
\newcommand{\candcpp}{\mbox{\tt C/C++}}

\newcommand{\gnu}{\mbox{\tt GNU}}
\newcommand{\gmp}{\mbox{\tt GMP}}
\newcommand{\gpp}{\mbox{\tt g++}}
\newcommand{\gppandgcc}{\mbox{\tt g++/gcc}}

\newcommand{\installpath}{\mbox{\tt \$\{INSTALL\_PATH\}}}
% \newcommand{\corepath}{\mbox{\tt \$\{INSTALL\_PATH\}/core\_vX.Y/}}
\newcommand{\corepath}{\mbox{\tt \$\{CORE\_PATH\}}}
\newcommand{\coredistfile}{\mbox{\tt core\_vX.Y.Z.tgz}}
\newcommand{\docdir}{\mbox{\tt \$\{CORE\_PATH\}/doc}}
\newcommand{\extdir}{\mbox{\tt \$\{CORE\_PATH\}/ext}}
\newcommand{\examplesdir}{\mbox{\tt \$\{CORE\_PATH\}/progs}}
\newcommand{\progsdir}{\mbox{\tt \$\{CORE\_PATH\}/progs}}
\newcommand{\includedir}{\mbox{\tt \$\{CORE\_PATH\}/inc}}
\newcommand{\includegmpdir}{\mbox{\tt \$\{CORE\_PATH\}/gmp/include}}
\newcommand{\libdir}{\mbox{\tt \$\{CORE\_PATH\}/lib}}
\newcommand{\gmplibdir}{\mbox{\tt \$\{CORE\_PATH\}/gmp/lib}}
\newcommand{\win}{\mbox{\tt \$\{CORE\_PATH\}/win32}}
% no longer used:
	% \newcommand{\lidiadir}{\mbox{\tt \$\{CORE\_PATH\}/lidia}}
\newcommand{\srcdir}{\mbox{\tt \$\{CORE\_PATH\}/src}}
\newcommand{\tempdir}{\mbox{\tt \$\{CORE\_PATH\}/tmp}}

\newcommand{\gmplink}{\mbox{\tt \$\{CORE\_PATH\}/gmp}}

\newcommand{\val}{\mbox{\rm Val}}
\newcommand{\err}{\mbox{\rm Err}}
\newcommand{\preci}{\mbox{\rm Prec}}
\newcommand{\appr}{\mbox{\rm Approx}}

\newenvironment{prog}{\begin{tabbing}
xx\=xxxx\=xxxx\=xxxx\=xxxx\=xxxx\=xxxx\=xxxx\=\kill \\}
{\end{tabbing}}

\newenvironment{progb}[1]{
\vspace{+\abovedisplayskip}
\fbox{
\begin{minipage}{.9\textwidth}
\vspace{-\abovedisplayskip}
\begin{prog}#1\end{prog}
\vspace{-\abovedisplayskip}
\end{minipage}
}
\vspace{+\abovedisplayskip}
}

\newcommand{\draftdimensions}{
        \setlength{\oddsidemargin}{-1cm}
        \setlength{\evensidemargin}{-1cm}
% \setlength{\textwidth}{1.2\textwidth}
        \setlength{\textwidth}{7in}
% \setlength{\textheight}{1.2\textheight}
        \setlength{\textheight}{9in}
        \setlength{\topmargin}{-5mm}
        \setlength{\parindent}{0mm}
        \setlength{\parskip}{5mm}
}



% \textwidth 15.4cm 
% \textheight 23.4cm
% \textheight 23cm
% \topmargin -14mm       
% \evensidemargin 3mm 
% \oddsidemargin 3mm

\draftdimensions

\title{\corelib\  Tutorial}

\author{Chen Li, Chee Yap, Sylvain Pion, Zilin Du and Vikram Sharma\\
Department of Computer Science\\
Courant Institute of Mathematical Sciences\\
New York University\\
%251 Mercer Street\\
New York, NY 10012, USA
}

\date{Nov 12, 2004\footnotemark[0]{\dag}
\footnotetext[0]{\dag
Revised: Jan 18, 1999;
Sep 9, 1999;
Aug 15, 2000;
Sep 1, 2001;
Jul 29, 2002;
Jun 20, 2003;
Nov 12, 2004.
This work has been funded by NSF Grants \#CCR-9402464,
\#CCF-0430836, and NSF/ITR Grant \#CCR-0082056.
}
}

\begin{document}

\maketitle

\abstract{
The \corelib\ is a collection of \cpp\ classes to support
numerical computations that have a variety of
precision requirements.  
In particular, it supports the Exact Geometric
Computation (EGC) approach to robust algorithms.
The implementation embodies our precision-driven
approach to EGC.  The library is designed to be
extremely easy to use.  Any \cpp\ programmer
can immediately transform a ``typical''
geometric application program into fully robust code,
without needing to transform the underlying program logic.
This tutorial gives an overview of the \corelib,
and basic instructions for using it.
}

%\newpage

%\section*{Table of Contents}

\vspace{.2in}
\begin{center}
\begin{minipage}{4in}
% \section*{Table of Contents}
  \begin{tabular}{c l c }
    {\bf Section} & {\bf Contents} & {\bf Page} \\
    1	& Introduction		& 2 \\
    2	& Getting Started	& 3 \\
    3	& Expressions		& 8 \\
    4	& Numerical Precision and Input-Output	& 9 \\
    5	& Polynomials and Algebraic Numbers & 15 \\
    6	& Converting Existing \candcpp\ Programs	& 16 \\
    7   & Using \core\ with \cgal\ & 19 \\
    8   & Efficiency Issues 	& 19 \\
    9	& \corelib\ Extensions	& 23 \\
%   X   & Implementation Issues and Features & ?? \\
    10	& Miscellany		& 23 \\
    11	& Bugs and Future Work	& 24 \\
    Appendix A & \core\ Classes Reference & 25 \\
    Appendix B & Sample Program		  & 49 \\
    Appendix C & Brief History		  & 50 \\
    	& References			  & 51 \\
   \end{tabular}
\end{minipage}
\end{center}

\cleardoublepage

\section{Introduction}

In programs such as found in
engineering and scientific applications, one can often
identify numerical variables that require more
precision than is available under machine arithmetic\footnote{
In current computers, this may be identified with
the IEEE 754 Standard.
}.
But one is also likely to find
other variables that need no more than machine precision.
E.g., integer variables used as array indices or for loop control.
The \corelib\ is a collection of \cpp\ classes
to facilitate numerical computation
that desire access to a variety of such precision requirements.
Indeed, the library even supports variables with irrational
values (e.g., $\sqrt{2}$) and allows exact comparisons with them.

Numerical non-robustness of programs is a widespread phenomenon,
and is clearly related to precision issues.
Two recent surveys are \cite{schirra:robustness-survey:98,yap:crc}.
Non-robustness is particularly insidious in geometric computation.
What distinguishes ``geometric computation'' from
general ``numerical computation'' is the appearance
of discrete or combinatorial structures, and the
need to maintain consistency requirements between
the numerical values and these structures \cite{yap:crc}.
Our library was originally designed to
support the {\em Exact Geometric Computation} (EGC)
approach to robust geometric computation
\cite{yap:exact,yap-dube:paradigm}.  The EGC approach
is one of the many routes that researchers have taken
towards addressing non-robustness in geometric computation.
Recent research in the computational geometry
community has shown the effectiveness
of EGC in specific algorithms such as
convex hulls, Delaunay triangulation, Voronoi diagram, mesh generation, etc
\cite{fortune-vanwyk:static:96,fortune-vanwyk:exact,kln:delaunay:91,bkmnsu:exact:95,burnikel:exact:thesis,shewchuk:adaptive:96}.
But programmers cannot easily produce such robust programs without
considerable effort.
A basic goal of our project is to create
a tool that makes EGC techniques accessible to \dt{all} programmers.
Through the \corelib, any \candcpp\ programmer can now
create robust geometric programs
{\em without} any special knowledge of EGC or other robustness techniques.
The \corelib, because of its unique
numerical capabilities, has other applications beyond EGC.
An example is in automatic theorem proving in geometry
\cite{tyl:zero-test:00}.

A cornerstone of our approach is 
to define a simple and yet natural numerical accuracy
API (Application Program Interface).
The \corelib\ defines four {\em accuracy levels}
to meet a user's needs: 

\begin{description}
\item[Machine Accuracy (Level 1)]
	This may be identified with the IEEE 
	Floating-Point Standard 754.
\item[Arbitrary Accuracy (Level 2)]
	Users can specify any desired 
	accuracy in term of the number of bits used in the computation. 
	E.g., ``200 bits'' means that the numerical operations will not 
	cause an overflow or underflow until 200 bits are exceeded.
\item[Guaranteed Accuracy (Level 3)]
	Users can specify the absolute 
	or relative precision that is guaranteed to be correct in the 
	final results. E.g., ``200 relative bits'' means that the first 
	200 significant bits of a computed quantity are correct.
\item[Mixed Accuracy (Level 4)]
	Users can freely intermix the various precisions at the level
	of individual variables.  This level is not fully defined,
	and only a primitive form is currently implemented.
\end{description}

Level 3 is the most interesting, and constitute the critical
capability of EGC.  Level 2 is essentially the
capability found in big number package, and in
computer algebra systems such as \texttt{Maple} or \texttt{Mathematica}.
There is a fundamental gap between Levels 2 and 3 that may not
be apparent to the casual user.

One design principle in our library is that a
\core\ program should be able to compile and run at
any of the four accuracy levels.  We then say that the program
can ``simultaneously'' access the different levels.
The current library development has focused mostly\footnote{
Level 1 effort simply amounts to ensuring
that a Level 3 program can
run at Level 1 as well.  A ``Level 3 program'' is one
that explicitly use classes or functions
that are specific to Level 3.
}
on Levels 1 and 3.  As a result of the simultaneous access
design, \core\ programs can be debugged and run at various levels
as convenient.  E.g., to test the general program logic, we
debug at Level 1, but to check the numerical computation,
we debug at Level 3, and finally, we may choose to run this
program at Level 2 for a speed/accuracy trade-off.

The mechanism for delivering these
accuracy levels to a program aims to be as transparent
as possible.  In the simplest situation,
the user begins with a ``standard'' \cpp\ program,
i.e., a \cpp\ program that does not refer to any 
\core-specific functions or classes.  We call this
a \dt{Level 1 program}.  Then the user
can invoke \corelib's numerical capabilities just
by inserting the line {\tt \#include "CORE/CORE.h"} into the
program, and compiling in the normal way.
In general, a key design objective is to reduce the effort
for the general programmer to write new robust programs,
or to convert existing non-robust programs into robust ones.

It should be evident that if an ``ordinary'' \cpp\ program
is to access an accuracy level greater than 1,
its basic number types must be re-interpreted and overloading
of arithmetic operators must be used.  In Level 2, the
primitive types \double\ and \lng\ are re-interpreted to refer
to the classes \BF\ and \Int, respectively.  Current
implementation encloses these values inside a number type \real.
In Level 3, both \double\ and \lng\ refer to the class \expr.
Think of an instance of the
\expr\ class as a real number which supports exact (error-less)
operations with $+,-,\times, \div$ and $\sqrt{}$,
and also exact comparisons.
Each instance of \expr\ maintains an \dt{approximate value}
as well as a \dt{precision}.   The precision is an
upper bound on the error in the approximate value.
Users can freely modify this precision, and
the approximate value will automatically adjust itself.
When we output an \expr\ instance, the current approximate
value is printed.

Our work is built upon the \rexpr\ Package
of Yap, Dub\'e and Ouchi \cite{yap-dube:paradigm}.
The \rexpr\ Package was the
first system to achieve Level 3 accuracy
in a general class of non-rational expressions.
The most visible change in the transition to \corelib\
is our new emphasis on ease-of-use.
The \core\ accuracy API was first proposed by
Yap \cite{yap:brown-cgc:98}.  An initial implementation
was described by Karamcheti et al \cite{klpy:core:98}.
At about the same time, Burnikel et al \cite{bfms:easy:99}
introduced the {\tt leda\_real} Library that is
very similar to Level 3 of our library.

The library has been extensively tested
on the Sun UltraSPARC, Intel/Linux and Windows platforms.
The main compiler for development is \gnu's \gpp.
% version 1.2 was tested using Sun's WorkShop Compiler {\tt CC}
% and SGI's MIPSpro Compiler {\tt CC}.
The base distribution for Version \versionNo\ is less than 800 KB, 
including source, extensions and examples.
The full distribution, which includes documentation and \gmp,
is less than 4MB.
It can be freely downloaded from our project homepage
	\begin{center}
	\verb+http://cs.nyu.edu/exact/core.+
	\end{center}
This tutorial has been updated for \corelib, Version \versionNo,
released on \releaseDate.

\section{Getting Started}

\paragraph{Installing the \corelib.}
% Software you need are \texttt{gunzip}, \texttt{tar},
% \texttt{make} and a \cpp\ compiler.
The \core\ distribution file is called
\coredistfile, where {\tt X.Y.Z} denotes the library version.
% coredistfile = core_vX.Y.Z.tgz.  
Thus, for the initial version \versionNo, we have {\tt X.Y.Z = \versionNo.0}.
Assume that the distribution file has been downloaded into
some directory \installpath.  In Unix, you can extract
the files as follows:
\\
\hspace*{1in}  {\tt \% cd} \installpath
\\
\hspace*{1in}  {\tt \% gzip -cd} \coredistfile {\tt\ | tar xvf -}
\\
where {\tt\%} is the Unix prompt.
This creates the directory {\tt core\_vX.Y} containing
all the directories and files.  Let \corepath\
% \corepath = $(CORE\_PATH)
be the full path name of this newly created directory:
thus \corepath\ expands to \installpath{\tt /core\_vX.Y}.
The \corelib\ directory structure is as follows:

\begin{tabbing}
XX\=XXXXXXXXXXXXXXX\=XXXXXX\=XXXXXX \kill\\
	\> \docdir: \> Documentation\\
	\> \includedir: \> The header files\\
	\> \srcdir: \> Source code for the \corelib\\
	\> \libdir: \> The compiled libraries are found here\\
	\> \extdir: \> Extensions for linear algebra and geometry, etc\\
	\> \examplesdir: \> Demo programs using the \corelib \\ 
% \examplesdir\ and \progsdir\ are same!
	\> \tempdir: \> Temporary directory \\ 
	\> \win: \> Director for Windows files \\ 
	\> \gmplink: \> gmp installation directory (may be a link)\\ 
\end{tabbing}
\noindent
The link \gmplink\ is not present after unpacking, but will be created
in the first three steps of the installation below.
The {\tt README} file in \corepath\ describes
the easy steps to compile the library, which are as follows:
\\
\hspace*{1in} \verb+ % cd ${CORE_PATH} +
\\
\hspace*{1in} \verb+ % make first       // determine system configurations for gmp+
\\
\hspace*{1in} \verb+ % make second      // make gmp libraries+
\\
\hspace*{1in} \verb+ % make third       // install gmp +
\\
\hspace*{1in} \verb+ % make testgmp     // check if gmp is properly installed +
\\
\hspace*{1in} \verb+ % make fourth      // make core library, extensionns, demo programs +
\\
\hspace*{1in} \verb+ % make fifth       // run sample programs +
\\
These steps assume that you downloaded the
full distribution (with \gmp) works in a unix-like environment,
including cygwin.  Variant installations
(e.g., for a base distribution, without \gmp)
are described in the {\tt README} file.
These five steps are equivalent to a simple ``make all''.
The first make will determine the system
configurations (platform, compilers, available files, etc).  
This information is needed for building the \gmp\ library,
which is the object of the second make.
These first two makes are the most expensive of the
installation, taking between 10--30 minutes depending on
the speed of your machine.  But you
can skip these steps in subsequent updates or
recompilation of the Core Library.
The third make will install the gmp library.
Before the fourth make, we do a simple check
to see if gmp has been properly installed (``make testgmp'').
The fourth make is equivalent to three separate makes,
corresponding to the following targets: {\tt corelib, corex, demo}.
Making {\tt corelib} creates the core library,
that is, it compiles the files in \srcdir\ resulting in the
file {\tt libcore++.a} which is then placed in \libdir.
Make {\tt corex} creates the Core Library extensions (\corex's),
resulting in the files {\tt libcorex++\_level*.a} being
placed in \libdir.  Note that we currently create
levels 1, 2 and 3 of the \corex.  Make {\tt demo} 
will compile all the sample programs in \examplesdir.
The fifth make will test all the sample programs.
The screen output of all the above makes are stored in
corresponding files in \tempdir.
An optional ``make sixth'' will run the fifth test with a single
timing number.  

\paragraph{Programming with the \corelib.}
\label{sec-prog-core}
It is simple to use the \corelib\ in your \candcpp\ programs. 
There are many sample programs and Makefiles under \examplesdir.
These could be easily modified to compile your own programs.
A simple scenario is when you already have
a working \cpp\ program which needs to be converted to
a CORE program.  The following 2 steps may suffice:

\begin{enumerate}
\item Modifying your program:
add one or two instructions as preamble to your program.
First, use a define statement to set the \core\ accuracy level:
	\begin{verbatim}
   #define CORE_LEVEL <level_number> // this line can be omitted when
                                     // using the default value 3.
	\end{verbatim}
Here {\tt <level\_number>} can be 1, 2, 3 or  4.
Next, include the \corelib\ header file {\tt CORE.h}
(found in \includedir)
	\begin{verbatim}
   #include "CORE/CORE.h"
	\end{verbatim}
To avoid potential name conflict, all header files are stored under
\includedir/CORE\footnote{
	Before \corelib\ 1.6, header files were in \includedir. 
	For backward compatibility, you still can use
	\texttt{\#include "CORE.h"}
}
This include line should appear {\em before} your code which
utilizes Core Library arithmetic, 
but {\em after} any needed the standard header files, e.g. \verb+<fstream>+.  
Note that \texttt{CORE.h} already includes the following:
\begin{verbatim}
     <cstdlib>, <cstdio>, <cmath>, <cfloat>, <cassert>, <cctype>,
     <climits>, <iostream>, <iomanip>, <sstream>, <string>.
\end{verbatim}

\item Quick start to compiling and running your own programs.
When compiling, make sure that \includedir\ and \includegmpdir\ are among the
include paths (specified by the {\tt -I} compiler flag)
for compiling the source.
When linking, you must specify the libraries from
\core\ and \gmp\ and the standard math library {\tt m},
using the {\tt -l} flag.  You also need to use the {\tt -L} flag
to place \libdir\ and \gmplibdir\ among the library paths.
E.g., to compile the program \texttt{foo.cpp}, type:
\\ \hspace*{0.2in}
 {\small\tt \% g++ -c -I\includedir\  -I\includegmpdir\ foo.cpp -o foo.o}
\\ \hspace*{0.2in}
 {\small\tt \% g++ -o foo -L\libdir\  -L\gmplibdir\ -lcore++ -lgmp -lm }
\\ in this order.

\end{enumerate}

The easy way to use the Core Library is to take advantage 
of the Core Library directory structure.  This can be seen
in how we compile all the demo programs.
First, create your own directory under \examplesdir\ and
put your program \texttt{foo.cpp} there.  Then copy one of the
Makefiles in \examplesdir.  E.g.,
\examplesdir{\tt /generic/Makefile}.
You can modify this make file to suit your needs.  To compile
\texttt{foo.cpp}, just modify the Makefile by adding the
following line as a new target:
	\begin{verbatim}
	foo: foo.o
	\end{verbatim}
To compile your program, you simple type ``make foo'' in this directory.
% See the \texttt{README} file for instructions if you use 
% dynamic libraries.  
The examples in this tutorial are found
in  \examplesdir\texttt{/tutorial/}. 

\paragraph{Namespace CORE.}
%Starting with CORE version 1.5, 
The library uses its own namespace called {\tt CORE}.
Therefore classes and functions of {\tt CORE}
are accessible by explicitly
prefixing them by {\tt CORE::}. E.g., {\tt CORE::Expr}.
You can also use the global statement~:

\begin{progb} {
\> \tt using namespace CORE;
} \end{progb}

In fact, this is automatically added by the include file
{\tt CORE.h} unless the compilation flag
\\ {\tt CORE\_NO\_AUTOMATIC\_NAMESPACE} is defined.

\paragraph{Basic Numerical Input-Output.}
Input and output of literal numbers come in three basic formats:
scientific format (as in {\tt 1234e-2}), positional format
(as in {\tt 12.34}), or rational format (as in {\tt 1234/100}).
Scientific and positional can be mixed (as in {\tt 1.234e-1})
and will be collectively known as the ``approximate number format''.
It is recognized by the presence of an ``e'' or a radical point.
In contrast, the rational format is known as ``exact number format'',
and is indicated by the presence of a ``/''.
For I/O purposes, a plain integer $1234$ is regarded 
as a special case of the rational format.  
Input and output of exact numbers is
pretty straightforward, but I/O of approximate numbers
can be subtle.

For output of approximate numbers, users can choose either
scientific or positional format, by calling the methods
{\tt setScientificFormat()} or
{\tt setPositionalFormat()}, respectively.
The output precision is manipulated 
using the standard \cpp\ stream manipulator {\tt setprecision(int)}.
The initial default is equivalent to
	\begin{verbatim}
   setprecision(6); setPositionalFormat();
	\end{verbatim}
Note that the term ``precision'' in \cpp\ streams is
not consistent with our use of the term (see Section 4).
In our terminology, precision are measured in ``bits'', while
input or output representation are in ``decimal digits''. 
In contrast, precision in \cpp\ streams
refers to decimal digits.
 
\begin{center}
\begin{tabular}{c}
\begin{progb}{
\>\tt \expr\ e1 = 12.34;          // constructor from C++ literals
\\
\>\tt \expr\ e = "1234.567890";  // constructor from string
\\ 
\>\>\>\tt // The precision for reading inputs is controlled by \definput
\\
\>\>\>\tt // This value is initialized to be 16 (digits).
\\
\>\tt cout << e << endl;
\\
\>\>\>\tt // prints 1234.57 as the output precision defaults to 6.
\\
\>\tt cout << setprecision(10) << e << endl; // prints 1234.567890
\\
\>\tt cout << setprecision(11) << e << endl; // prints 1234.5678900
\\
\>\tt setScientificFormat();
\\
\>\tt cout << setprecision(6) << e << endl;  // prints 1.23457e+4 
}\end{progb}
\end{tabular}
Program 1
\end{center}

Program 1 uses \expr\ for illustrating the main issues
with input and output of numerical values.
But all CORE number classes will accept string inputs as well.
Of course, in Level 3, \expr\ will be our main number type.

For input of approximate numbers, two issues arise.
As seen in Program 1,
expression constructors accept two kinds of literal number inputs:
either standard \cpp\ number literals (e.g., {\tt 12.34}, without any quotes)
or strings (e.g., {\tt "1234.567890"}).
You should understand that the former is inherently inexact:
E.g., {\tt 12.34} has no exact machine double representation,
and you are relying on the compiler to convert this into machine precision.
Integers numbers with too many digits (more than $16$ digits)
cannot be represented by \cpp\ literals.
So users should use string inputs to ensure full
control over input precision, to be described next. 

Yet another issue is the base of the underlying
natural numbers, in any of the three formats.   
Usually, we assume base $10$ (decimal representation).
However, for file I/O (see below) of large numbers,
it is important to allow non-decimal representations.

I/O streams understand expressions (\expr).  The output precision in both
scientific and positional formats is equal 
to the number of digits which is printed, provided
there are that many correct digits to be printed.
This digit count does not include the decimal point, the ``\texttt{e}''
indicator or the exponent value in scientific format.
But the single $0$ before a decimal point is counted.

In two situations, we may print less than the maximum possible:
(a) when the approximate value of the expression does not have
that many digits of precision, and
(b) when the exact output does not need that many digits.
In any case, all the output digits are correct except that
the last digit may be off by $\pm 1$.  Note that 
19.999 or 20.001 are considered correct outputs for the value 20,
according to our convention.  But 19.998 or 20.002 would not qualify.
Of course, the approximate value of the expression can be
improved to as many significant digits as we want -- we simply
have to force a re-evaluation to the desired precision before output.

Three simple facts come into play
when reading an approximate number format into internal representation:
(1) We normally prefer to use floating point in our internal
representation, for efficiency.
(2) Not all approximate number formats can be exactly represented by
a floating point representation when there is a change of base.
(3) Approximate number format can always be represented exactly
by a big rational.

We use the value of the global variable \definput\ to
determine the precision for reading literal input numbers.
This variable can be set by
calling the function\\
\verb+setDefaultInputDigits(+\extlong\verb+)+.
Here, the class \extlong\ is basically a wrapper around the machine
\lng\ type which supports special values such as $+\infty$, denoted
by \coreInfty.  If the value of \definput\ is $+\infty$, then the
literal number is internally represented without error,
as a rational number if necessary.  If \definput\ is a finite integer $m$,
then we convert the input string to a \BF\ whose
value has absolute error at most $10^{-m}$, which in base $10$
means the mantissa has $m$ digits.  The initial default value of
\definput\ is $16$.

\paragraph{A Simple Example.}
\label{sec-example-sqrt}
Consider a simple program to compare the following
two expressions, numerically:
	$$\sqrt{x}+\sqrt{y} : \sqrt{x+y+2\sqrt{xy}}.$$
Of course, these expressions are algebraically identical, and
hence the comparison should result in equality regardless
of the values of $x$ and $y$.  Running the following program
in level 1 will yield incorrect results, while level 3 is always correct.

\begin{center}
\begin{tabular}{c}
\begin{progb}{
\> \tt \#ifndef CORE\_LEVEL\\
\> \tt \#   define CORE\_LEVEL 3\\
\> \tt \#endif\\

\> \tt \#include "CORE/CORE.h"  // this must come after the standard headers\\

\> \tt int main() \{\\
\>\>  \tt setDefaultInputDigits(CORE\_INFTY);\\
\>\>  \tt double x = "12345/6789"; \ \ \ \ \  \  // rational format\\
\>\>  \tt double y = "1234567890.0987654321"; \ \ \ \ \ //  approximate format\\
\\
\>\>  \tt double e = sqrt(x) + sqrt(y);\\
\>\>  \tt double f = sqrt(x + y + 2 * sqrt(x*y));\\
\\
\>\>  \tt std::cout << "e == f ? " << ((e == f) ? \\
\>\>\>\tt 	  "yes (CORRECT!)" :\\
\>\>\>\tt 	  "no (ERROR!)"  ) << std::endl;\\
\> \}
}\end{progb}
\end{tabular}
Program 2
\end{center}

\ignore{
In Karamcheti et al \cite{klpy:core:98}, a
simple experiment of this kind was described.
A straightforward program which intersects a pair of lines and
}

\paragraph{Terminology.}
We use the capitalized ``CORE''
as a shorthand for ``Core Library'' (e.g., a CORE program).
Note that ``Core'' is not an abbreviation; we chose this name 
to suggest its role as the ``numerical core''
for robust geometric computations.
%It also reminds us that ``cores'' are usually small, not bloated.  

\section{Expressions}

The most interesting part of the \corelib\ is its
notion of expressions, embodied in the class \expr.
This is built on top of the class
\real, which provides a uniform
interface to the following subtypes of real numbers:
\\
\hspace*{1in} \int, \lng, \float, \double, \Int, \Rat, \BF.  
\\
Instances of the class \expr\ can be thought of as
algebraic expressions built up from instances of \real\
and also real algebraic numbers, via the operators
$ +, -, \times, \div$ and $\sqrt{\phantom{\cdot}}$.
Among the subtypes of \real, the
class \BF, has a unique and key role in our system,
as the provider of approximate values.
It has an additional property not found
in the other \real\ subtypes, namely, each \BF\ keeps
track of an error bound, as explained next.

The simplest use of our library is to avoid 
explicit references to these classes (\expr, \real, \BF).
Instead use standard \cpp\ number types (\int,\lng,\float,\double)
and run the program in level 3.
Nevertheless, advanced users may find it useful to
directly program with our number types.
Appendix A serves as a reference for these classes.

\paragraph{Level 2 and Level 3 numbers.}
There are two inter-related concepts in
the \corelib: {\em precision} and {\em error}.
One may view them as two sides of the same coin --
a half-empty cup versus a half-full cup.
Within our library, they are used in a technical
sense with very different meanings.
Let $f$ be an instance of the class \BF,
and $e$ be an instance of the class \expr.
We call $f$ a ``Level 2 number'' and $e$ a ``Level 3 number''.  
Basically, we can compute with Level 2 numbers\footnote{
%
Level 2 numbers ought to refer to any
instance of the class \real, if only they all
track an error bound as in the case of the \BF\ subtype.
Future implementations may have this property.
%
} to any desired error bound.  But unlike Level 3 numbers,
Level 2 numbers cannot guarantee error-less results.
The instance $f$ has a \dt{nominal value} $\val_f$ (a real number)
as well as an \dt{error bound} $\err_f$.
One should interpret $\err_f$ as an upper bound on the difference
between $\val_f$ and some ``actual value''.  The instance $f$ does not
know the ``actual value'', so one should just view $f$ as the
interval $\val_f \pm \err_f$.  But a
user of Level 2 numbers may be able to keep track of this ``actual value''
and thus use the interval properly.  Indeed, Level 3 numbers
uses Level 2 numbers in this way.

The idea of Level 3 numbers is novel, and
was first introduced in our \rexpr\ package \cite{yap-dube:paradigm}.
The expression $e$ also has a \dt{value}
$\val_e$ which is exact\footnote{
And real, for that matter.
}.  Unfortunately, the value $\val_e$ is in the mathematical realm ($\RR$)
and not directly accessible.
Hence we associate with $e$ two other quantities:
a \dt{precision bound} $\preci_e$ and an \dt{approximation} $\appr_e$.
The library guarantees that the approximation error $|\appr_e - \val_e|$
is within the bound $\preci_e$.
The nature of $\preci_e$ will be explained in the next section.
What is important is that $\preci_e$ can be freely 
set by the user, but the approximation $\appr_e$
is automatically computed by the system.
In particular, if we increase the precision
$\preci_e$, then the approximation $\appr_e$ will be
automatically updated if necessary.

In contrast to $\preci_e$, the error bound $\err_f$
should not be freely changed by the user.  This is because
the error is determined by the way that $f$ was derived, and must
satisfy certain constraints (basically it is the constraints
of interval arithmetic).
For instance, if $f=f_1+f_2$ then the error bound in $f$ is
essentially\footnote{
In operations such as division or square roots, exact results
may not be possible even if the operands
have no error.  In this case, we rely on some global parameter to
bound the error in the result.
}
determined by the error bounds in $f_1$ and $f_2$.
Thus, we say that error bounds are
% {\em \'a posteriori values}
{\em a posteriori values}
while precision bounds are
% {\em \'a priori values.}
{\em a priori values.}

\paragraph{Error-less Comparisons.}
While we can generate arbitrarily accurate approximations to
a Level 3 number, 
this does not in itself allow us to do exact comparisons.
When we compare two numbers that happen to be equal,
generating increasingly accurate approximations can
only increase our confidence that they are equal, but
never tell us that they are really equal.
Thus, there is a fundamental gap between Level 2 and Level 3 numbers.
To be able to tell when two Level 3 numbers are equal,
we need some elementary theory
of algebraic root bounds \cite{yap:algebra-bk}.
This is the basis of the Exact Geometric Computation (EGC) approach
to non-robustness. 

\section{Numerical Precision and Input-Output}

Numerical input and output may be subtle,
but they should never be ambiguous in our system.
A user should know how input numbers are read,
but also know how to interpret the output of numbers.

For instance, confusion may arise from the fact that
a value may be {\em exactly} represented internally,
even though its printout is generally an approximation.
Thus, the exact representation of $\sqrt{2}$
is available internally in some form,
but no printout of its approximate numerical value can tell you
that it is really $\sqrt{2}$.  For this, you need to
do a comparison test.

NOTE: Precision is always given in base $10$ or base $2$.  
Generally, we use base $2$ for internal precision, and use base $10$ for
I/O precision.  \core\ has various global variables
such as \defabs\ and \definput\ that controls precision in one
way or other.   Our naming convention for such variables tells you
which base is used:
precision variables in base $10$ have the substring \texttt{Digit},
while precision variables in base $2$ have the substring \texttt{Prec}.

\paragraph{The Class of Extended Longs.}
For programming, we introduce
an utility class called \extlong\ (extended long)
which is useful for expressing various bounds.
In our system, they are used not only for specifying
precision bounds, but also root bounds as well.
Informally, \extlong\ can be viewed as a wrapper
around machine \lng\ and which supports the
special values of $+\infty, -\infty$ and NaN (``Not-a-Number'').
These values are named \posInfty, \negInfty\ and \NaN, respectively.
For convenience, \coreInfty\ is defined to be \posInfty.
The four arithmetic operations on extended
longs will never lead to exceptions such as overflows or divide-by-zero\footnote{
%
To delay the onset of overflows, it may be useful to extend
\extlong\ to implement a form of level arithmetic.
E.g., when a value overflows machine \lng, we can keep track
of $\log_2$ of its magnitude, etc.
%
} or undefined values.  This is because such operations can
be detected and given the above special values.
A user may use and assign to \extlong's just as they would to machine \lng's.

\paragraph{Relative and Absolute Precision.}
Given a real number $ X $, and integers $a$ and $r$, we say that a
real number $ \widetilde{X} $ is an
{\em approximation} of $X$ to {\em (composite) precision}
$ [r, a] $, denoted
\[
\widetilde{X} \simeq X \left[ r, a \right],
\]
provided either
\[
\left| \widetilde{X} - X \right| \le 2^{-r}\left| X \right|
	\qquad \mbox{or} \qquad
\left| \widetilde{X} - X \right| \le 2^{-a}.
\]
Intuitively, $r$ and $a$ bound the number of ``bits'' of relative and
absolute error (respectively) when $\widetilde{X}$ is used to approximate $X$.
Note that we use\footnote{
Jerry Schwarz, ``A C++ library for infinite precision
floating point'' (Proc.~USENIX C++ Conference, pp.271--281, 1988)
uses the alternative ``and'' semantics.
} the
``or'' semantics (either the absolute ``or'' relative error has the
indicated bound).  In the above notation, we view the combination
``$X[r,a]$'' as the given data (although $X$ is really a black-box,
not an explicit number representation)
from which our system is able to generate an approximation $\widetilde{X}$.
For any given data $X[r,a]$, we are either in the ``absolute regime''
(if $2^{-a}\ge 2^{-r}|X|$) or in the ``relative regime''
(if $2^{-a}\le 2^{-r}|X|$).

To force a relative precision of $ r $, we can
specify $a = \infty $.  Thus $ X[r, \infty]$ denotes any
$\widetilde{X}$ which satisfies
$\left| \widetilde{X} - X \right| \le 2^{-r}\left| X \right|$.
Likewise, if $ \widetilde{X} \simeq X[\infty, a] $ then $ \widetilde{X} $ is
an approximation of $ X $ to the absolute precision $a$,
$|\widetilde{X} - X| \le 2^{-a}$.

In implementation, $r$ and $a$ are \extlong\ values.
We use two global variables to specify the global composite precision:
	\begin{equation}\label{eq:defrel}
	[\defrel,\defabs].
	\end{equation}
It has the default value $[60,\coreInfty]$.
The user can change these values at run time by calling the functions:
\begin{verbatim}
     long setDefaultRelPrecision(extLong r);    // returns previous value
     long setDefaultAbsPrecision(extLong a);    // returns previous value
     void setDefaultPrecision(extLong r, extLong a);
\end{verbatim}

How does the default precision in \ref{eq:defrel} control your computation?
Say you perform arithmetic operations such as \texttt{z = x/y;}
The system will ensure that the computed value of
\texttt{z} satisfies the relation \texttt{z}$\sim x/y[\defrel,\defabs]$.

Sometimes, we want to control this precision for individual
variables.  If {\tt e} is an \expr, the user can invoke 
{\tt e.approx(rel, abs)} where {\tt rel, abs} are
extended longs representing the desired composite precision.
The returned value is a \real\ instance that satisfies this requested
precision.
If {\tt approx} is called without any arguments, it will
use the global values $[\defrel,\defabs]$.

In Section 2 (``Getting Started''),
we gave the basics for numerical input and output.
In particular, we have 3 formats:
positional (e.g., {\tt 3.14159}),
scientific (e.g., {\tt 314159 e-5}), or
rational (e.g., {\tt 314159/100000}).
These formats can be read as a \cpp\ literal or as a string.
But there are important differences related to precision.

\paragraph{Precision of Numerical Input.}
Consider the following input of numerical values:

   \begin{verbatim}
       Expr e = 0.123;      // position format in machine literal
       Expr f = "0.123";    // positional format in string
       Expr g = 123e-3;     // scientific format in machine literal
       Expr h = "123e-3";   // scientific format in string
       Expr i = 12.3e-2;    // mixed format in machine literal
       Expr j = "12.3e-2";  // mixed format in string
   \end{verbatim}
   
The input for expressions \texttt{e}, \texttt{g} and \texttt{i}
are \cpp\ number literals, and you may
expect some error when converted into the internal representation.
But the relative error of the internal representation is at most
$2^{-53}$, assuming the IEEE standard.  
In contrast, the values of the expressions
\texttt{f},\texttt{h} and \texttt{j}
are controlled by the global variable \definput.
If \definput\ has the value \coreInfty\ then
\texttt{f},\texttt{h} and \texttt{ j}
will have the exact rational value $123/1000$.
Otherwise, they will be
represented by \BF\ numbers whose absolute error is
at most $10^{- \tt m}$ where {\tt m = }\definput.

Instead of using constructors, we can also read input
numbers from streams.  E.g.,
	\begin{verbatim}
	Expr k;
	cin >> k;
	\end{verbatim}
In this case, the input number literals are regarded as strings,
and so the value of the variable \texttt{k} is controlled by \definput.

\paragraph{Precision of Numerical Output.}
Stream output of expressions is controlled by the
precision variable stored in the stream {\tt cout}.
Output will never print inaccurate digits,
but the last printed digit may be off by $\pm 1$.  
Thus, an output of $1.999$
may be valid when the exact value is $2$, $1.998$, $1.9988$ or $1.9998$.
But this output is invalid when the exact value is $1.99$ (since
the last digit in $1.999$ is misleading) or $2.01$.
Similarly,
an output of $1.234$ is invalid when the exact value is $1.2$ or $1.23$.

\paragraph{Output Number Formats.}
We have two formats for approximate numbers:
scientific and positional.  But
even when positional format is specified, under certain
circumstances, this may be automatically overridden,
and the scientific format used.  
For instance, if the output precision is $3$ and the number
is $0.0001$ then a positional output would be $0.00$.  In this
case, we will output in scientific format as {\tt 1.00e-4} instead.
Again, if the number is an integer $1234$, then we will output in
scientific format as {\tt 1.23e+3}.  
In both cases, we see why the positional output (restricted
to 3 digits) is inadequate and the scientific format (also
restricted to 3 digits) is more accurate.
See Appendix A.1.6 for details.

One issue in numerical output is how to tell the 
users whether there is any error in an output or not.
For instance, if you print the value of \texttt{Expr("1.0")},
you may see a plain \texttt{1.} (and not \texttt{1.0}).
It is our way of saying that this value is exact.
But if you print the value of \texttt{Expr(1.0)}, you may be surprised to
see \texttt{1.0000}.  Why the difference? Because in the former
case, the internal representation is a \Rat\ while
in the latter case is a machine \double.  The latter is
inherently imprecise and so we print as many digits as the
current output precision allows us (in this case 5 digits).
But in printing a \Rat\ we do not add terminal $0$'s.

\paragraph{Interaction of I/O parameters.}
It should be clear from the preceding
that the parameters \defrel, \defabs, \definput, and (stream) output precision
interact with each other in determining I/O behavior:

\begin{center}
\begin{tabular}{c}
\begin{progb}{
\> \tt setScientificFormat();
\\
\> \tt setDefaultInputDigits(2); \ \ \ \ \ \ \ \ \ \ \ \ // defInputDigits = 2
\\
\> \tt Expr X = "1234.567890";
\\
\> \tt cout << setprecision(6); \ \ \ \ \ \ \ \ \ \ \ \ \ // output precision = 6
\\
\> \tt cout << X << endl; \ \ \ \ \ \ \ \ \ \ \ \ \ \ \ \ // prints .123457e+4
\\
\> \tt cout << setprecision(10) << X << endl;  // prints .1234567871e+4
\\
\> \tt cout << setprecision(100) << X << endl;  // prints .123456787109375000e+4
}\end{progb}
\end{tabular}
	Program 3
\end{center}

Note that since the input precision is set to $2$,
the internal value of $X$ is an approximation to $1234.567890$ with
an error at most $10^{-2}$.  Thus the second output of $X$
printed some ``wrong digits''.  In particular, the output
$1234.567871$ contains $8$ correct digits.  
However, notice that our semantic guarantees
only 6 correct digits -- so this output has given us 2
``bonus digits''.  In general, it is difficult to predict how many
bonus digits we may get.
Our convention for expressions is that all leaves are error-free
and thus the output may appear strange (although it is not wrong).
In fact, if we set the output precision to $100$, we see that
expression $X$ is assigned the exact value of $1234.56787109375000$
(we know this because this output was terminated prematurely before
reaching 100 digits). 
To force exact input, you must set \definput\ to $+\infty$:

\begin{center}
\begin{tabular}{c}
\begin{progb}{
\> \tt setScientificFormat();
\\
\> \tt setDefaultInputDigits(\coreInfty);
\\
\> \tt Expr X = "1234.567890"; // exact input
\\
\> \tt cout << setprecision(6) << X << endl; \ \ // prints .123457e+4
\\
\> \tt cout << setprecision(10) << X << endl; \ // prints .1234567890e+4
\\
\> \tt cout << setprecision(100) << X << endl; // prints .1234567889999999999e+4
\\
\> \tt X.approx(CORE\_INFTY, 111); // enough for 33 digits.
\\
\> \tt cout << setprecision(33) << X << endl;
\\
\> \tt // prints 33 digits: .123456789000000000000000000000000e+4
}\end{progb}
\end{tabular}
	Program 4
\end{center}

% new for Core 1.4:

\paragraph{Output Stream Precision.}
Besides the output precision parameters for \BF,
the above examples illustrate yet another parameter
that controls output: namely the output precision parameter
that is associated with output streams such as \texttt{cout}.
This parameter as set using the standard \texttt{setprecision(int)} method
of output streams.   Core provides another way to do this,
which is currently not entirely 
consistent with \texttt{setprecision(int)} method.
%
% TO MAKE IT CONSISTENT, WE ONLY NEED TO
%	modify "setprecision(int)" to update \defOutputDig\.  
%
Namely, you can call the method
\texttt{setDefaultOutputDigits(long p)} method.  This method
will perform the equivalent of \texttt{setprecision(p)}, but in addition
updates the global parameter \defOutputDig\ to \texttt{p}.
It returns the previous value of \defOutputDig.
The initial default value of \defOutputDig\ is $10$.
% \defOutputDig\ is normally set equal to defBFOutputPrec.

\paragraph{Output for Exact and Inexact \BF\ Values.} 
A \BF\ is represented by a triple of integers
$(man, exp, err)$ where $man$ is the mantissa,
$exp$ the exponent and $err$ the error.  This represents
an interval $(man \pm err)\times B^{exp}$ where
$B=2^{-14}$ is the base.  When $err=0$, we say the
\BF\ value is \dt{exact} and otherwise \dt{inexact}.
For efficiency, we normalize error so that $0\le err\le B$.
Since we do not want to show erroneous digits in the
output, the presence of error ($err>0$) is important
for limiting the number of \BF\ output digits.
To illustrate this, suppose $B=2$ instead of $2^{14}$
and our \BF\ is $x = (man, exp, err)=(10, -5, 0)$,
written here with decimal integers.
Then \texttt{cout << setprecision(6) << $x$} will
show $0.3125$.  In general, we expect such a floating
point number to have an error $B^{exp}=2^{-5}=0.03125$.
Then $x = 0.3125 \pm 0.03125$.  This suggests that
we should not output beyond the first decimal place.
To ensure this behavior, we can simply set the
error component to $1$.  The method to call is
$x$.{\tt makeInexact()}, and as a result
$x$ becomes $(man, exp, err)=(10, -5, 1)$.  Now, 
\texttt{cout << setprecision(6) << $x$} will output
only $0.3$.  There is a counterpart $x$.{\tt makeExact()}
that makes $err=0$.  The Section on Efficiency Issues has
related information on this.

\paragraph{Connection to Real Input/Output.}
Expression constructor from strings
and stream input of expressions are derived
from the \real\ constructor from strings.
See Appendix A.2.1 and A.2.5 for more details.
On the other hand,
stream output of expressions is derived from
the output of \BF\ values.  
See Appendix A.1.6 for this.

\paragraph{String and File I/O.}
Instead of input/output from/to a stream,
we can also input/output from/to a string.
The methods are called \texttt{toString} and \texttt{fromString}.
These methods are available for the number classes \BF, \Int,
and \Rat.  For \real\ and \expr\ we only have the method
\texttt{toString}.  In place of \texttt{fromString}, you can
directly assign (=) a string to \real\ and \expr\ variables.

The directory \examplesdir\ contains 
examples of both string and file I/O.

The need to read very large numbers from files, and
to write them into files, is important for certain computations.
Moreover, base for representing these numbers in files
should be flexible enough to support standard bases for
big integers (base 2, 10 and 16).
Such a format is specified
in \corepath\texttt{/progs/fileIO}.  In particular, this format assumes that
files are ascii based, for flexibility and human
readability.  Four basic formats are defined:
	\begin{center}
	Integer, Float, Normalized Float (NFloat) and Rational.
	\end{center}
The following methods are available:

\begin{center}
\begin{tabular}{c}
\begin{progb}{
\\
\> \tt void BigInt::read\_from\_file(istream is, long maxLen = 0)
\\
\> \tt void BigInt::write\_to\_file(ostream os, int base = 10, long lineLen = 70)
\\
\> \tt void BigFloat::read\_from\_file(istream is, long maxLen = 0)
\\
\> \tt void BigFloat::write\_to\_file(ostream os,
	int base = 10, long lineLen = 70)
\\
\> \tt void BigRat::read\_from\_file(istream is, long maxLen = 0)
\\
\> \tt void BigRat::write\_to\_file(ostream os, int base = 10, long lineLen = 70)
}\end{progb}
\end{tabular}
\end{center}

There are two bases in the representation of a \BF, the base
of the mantissa and the base for raising the exponent.
We call these the ``mantissa base'' and the ``exponent base''.
We view \Int\ as having only a mantissa base.
In the above arguments, \texttt{base} is the mantissa base.
The exponent base is defaulted to the internal base representation
used by our \BF\ class (which is $2^{14}$).
Also {\tt maxLen} indicates the number of bits (not ``digits'' of the
input file base) to be read from the input file.  
The default value of {\tt maxLen = 0} means we read all the
available bits.  The {\tt lineLen} tells us how many digits should
be written in each line of the mantissa.
Note that when we read into a \BF\, and the base is 10, then
this may incur an approximation but the error is guaranteed
to be smaller than half a unit in the last digit.

\paragraph{Conversion to Machine Double.}
This is strictly speaking not a numerical I/O issue,
but one of inter-conversion among number types.
Such details are described under individual
number classes (Appendix A).  However, the conversion of our internal
numbers into machine double values is an important topic
that merits a place here.

All our number classes has a \texttt{doubleValue()} method
to convert its value to one of the two 
nearest machine representable double value.  
For instance, if \texttt{e} is an expression,
we can call \texttt{e.doubleValue()} a machine double value.
We do not guaranteed rounding to nearest but that
this value is one of the two closest machine doubles
(either ceiling or floor).  See \progsdir\texttt{testSqrt.cpp}
for some discussion of the conversion errors.
It is also useful to convert an exact value into
an interval defined by two machine doubles.
The method \texttt{Expr::doubleInterval()} does this.

It is important to realize that all these conversions
to machine doubles may overflow or underflow.
It is the user's to check for this possibility.
The function \texttt{int finite(double)} can be used for
this check: it returns a zero
if its argument does not represent a finite
double value (i.e., the argument is either NaN or infinity).

\section{Polynomials and Algebraic Numbers}
	\label{sec-algebraic}

Beginning in Version 1.6, we introduce arbitrary real algebraic
numbers into expressions.  For example, suppose we want to
define the golden ratio $\phi= 1.618033988749894\ldots$.
It can be defined as the unique positive root of the polynomial
$P(X)= X^2 - X - 1$. 
We have a templated (univariate) polynomial class
named {\tt Polynomial<NT>} where {\tt NT} is a suitable number type
to serve as the type of the coefficients in the polynomial.
We allow {\tt NT} to be \Int, \expr, \BF, \Rat, \lng\ or \int.
Some \texttt{Polynomial} methods may not be meaningful or obvious
for certain choices of {\tt NT}.  For instance, 
many polynomial operations such as polynomial GCD depend on the concept of
divisibility in {\tt NT}.  But the notion of divisibility for
\BF\ values may not be so obvious (but see Appendix A under \BF).
For \Rat, we may take the position that divisibility is the trivial relation
(every nonzero value can divide other values).  But this will not
be our definition.
For \lng\ and \int, clearly coefficients may overflow.
We regard \Int\ as the ``canonical'' number type for
polynomial coefficients because all polynomial methods
will be fairly clear in this case.  Hence, our
polynomial examples will usually take \texttt{NT}=\Int.

First consider how to input polynomials.
There is an easy way to input such a polynomial, just as the
string \texttt{"x\^{}2 - x - 1"}.  
Currently, the coefficients from string input are assumed
to be \Int.  A slower way (which may be more
appropriate for large polynomials) is 
to construct an array of its coefficients.  

\begin{center}
\begin{tabular}{c}
\begin{progb}{
\> \tt typedef BigInt NT;
\\
\> \tt typedef Polynomial<NT> PolyNT;
\\
\> \tt NT coeffs[] = \{-1, -1, 1\}; // coeffs[i] is the coefficient of $X^i$
\\
\> \tt PolyNT P(2, coeffs); \ \ \ \ \ // P = $X^2 - X - 1$
\\
\> \tt PolyNT Q = "x\^{}2-x-1"; \ \ \ \ // Q = $X^2 - X - 1$
}\end{progb}
\end{tabular}
\end{center}

We use the polynomial {\tt P} to define an \expr\ whose value is $\phi$.
There are two ways to identify $\phi$ as the intended root of {\tt P}:
we can say $\phi$ is the $i$-th smallest root of {\tt P} (where $i=2$)
or we can give a \BF\ interval {\tt I = [1, 2]} that contains
$\phi$ as the unique root of {\tt P}.   We may call the arguments
$i=2$ or {\tt I}$=[1.0,2.0]$ the \dt{root indicators} for the polynomial {\tt P}.
The other root of {\tt P} is $\phi' = 1-\phi$, and it has the
root indicators $i=1$ or {\tt I = [-1.0, 0.0]}.

\begin{center}
\begin{tabular}{c}
\begin{progb}{
\> \tt Expr phi1 = rootOf(P, 2); \ \  // phi1 is the 2nd smallest root of P
\\
\> \tt BFInterval I(-1.0, 0.0); \ \ \  // I is the interval [-1, 0]
\\
\> \tt Expr phi2 = rootOf(P, I); \ \  // phi2 is the unique negative root of P
\\
\> \tt Expr Phi2 = rootOf(P, -1, 0); // Alternative
}\end{progb}
\end{tabular}
\end{center}

To test that {\tt phi1} and {\tt phi2} have the correct values,
we can use the fact that $\phi+\phi' = 1$.   This is done in
the next code fragment.

Alternatively, we can use the fact that
$\phi$ and $\phi'$ are still quadratic numbers, and
we could have obtained these values by our original {\tt sqrt} operators,
e.g., $\phi= (1+\sqrt{5})/2$.   However, we offer a convenient
alternative way to get square roots, by using the radical operator.
In general, {\tt radical(n, m)} gives the {\tt m}-th root of the
number {\tt n}. Compared to the Version 1.6 the current version can handle
any positive number type (\Int, \expr, \BF, etc.) for $n$.

\begin{center}
\begin{tabular}{c}
\begin{progb}{
\> \tt if (phi1 + phi2 == 1) cout << "CORRECT!"  << endl;
\\
\> \tt else cout << "ERROR!" << endl;
\\
\> \tt Expr goldenRatio = (1 + radical(5,2))/2: //  another way to specify phi
\\
\> \tt if (phi1 == goldenRatio) cout << "CORRECT!" << endl;
\\
\> \tt else cout << "ERROR!" << endl;
}\end{progb}
\end{tabular}
	Program 5
\end{center}

Recall that the constructible reals are those real numbers
that can be obtained from the rational operations and the
square-root operation.
The new constructors {\tt rootOf}
and {\tt radical} can give rise to numbers which are
not constructible reals.  

It is possible to construct invalid \expr\ when we specify
an inappropriate root indicator:
when {\tt i} is larger than the number of real roots in the polynomial {\tt P},
or when {\tt I} contains no real roots or contains 
more than one real root of {\tt P}.
We can generalize this by allowing root 
indicators that are negative integers:
if {\tt i} is negative, we interpret this to refer to the
($-${\tt i})-th largest
real root.  E.g., if {\tt i} is $-2$, then we want the 2nd largest real
root.  Moreover, we allow {\tt i} to be $0$, and this refers to
the smallest {\em positive} root.   When the root indicator is completely
omitted, it defaults to this special case.

To support these methods, we define the Sturm class that
implements Sturm techniques and Newton iterations.
For more details, see Appendix A.4 (Polynomials) and Appendix A.5 (Sturm).

\paragraph{Beyond univariate polynomials.}
In Version 1.7, we introduce bivariate polynomials
as well as algebraic curves.  We provided
basic functionality for doing arithmetic on curves
as well as graphical display functions (based 
on openGL) are available.
These may be found under \progsdir\texttt{/curves}.

\section{Converting Existing \candcpp\ Programs}
	\label{sec-convert}

Most of the following rules are aimed at making
a Level 1 program compile and run at Level 3.
So, this section might also be entitled
``How to make your \candcpp\ program robust''.

\begin{enumerate}

\item
There is a \dt{fundamental rule} for writing
programs that intend to call the \corelib: all arithmetic
operations and comparisons should assume error-free results.
In other words, imagine that you are really operating
on members of the mathematical domain $\RR$ of real numbers,
and operations such as $+, -, \times, \div, \sqrt{}$ return
exact (error-free) results.
Unfortunately, programs in conventional
programming language that have been made
``numerically robust''  often apply tricks that violate this rule.
Chief among these tricks is \dt{epsilon-tweaking}.
Basically this means that all comparisons to zero
are replaced by comparison to some small, program-dependent
constant (``epsilon'').  There may be many such
``epsilons'' in the program.  This is a violation of our
fundamental rule.  

Perhaps the simplest way to take care
of this is to set these epsilons to $0$ when in Level 3.  
There is only one main concern here.
When comparing against such epsilons, most
programmers do not distinguish between ``$\le$'' and ``$<$''. 
Often, they exclusively use ``$<$'' or ``$>$'' in comparisons against
an epsilon.
E.g., ``$|x|< \varepsilon$'' is taken as equivalent to $x=0$.
If you now set $\varepsilon$ to $0$ in level 3, it is clear that
you will never succeed in this test.  Note that in \candcpp\ the
usual function for absolute value is \texttt{fabs(x)}.  This
function will also work correctly in Level 3. 

\item
In your code that follows the preamble

\begin{verbatim} 
        	#define CORE_LEVEL 3
        	#include "CORE/CORE.h" 
\end{verbatim} 
\noindent
you must remember that the built-in machine
types \double\ and \lng\ will be replaced by (i.e., promoted to)
the class \expr.  An analogous promotion
occurs at Level 2.  If you do not want such promotions to
occur, be sure to use {\tt machine\_double} and {\tt machine\_long}
instead of \double\ and \lng, respectively.

If you are including a standard library, it is important to ensure
that such promotions are not applied to your library.
An example is the standard \cpp\ library \texttt{<fstream>},
when you need to perform file I/O in a \core\ program.
Therefore such libraries should be placed before the
inclusion of \texttt{CORE.h}.

\item
All objects implicitly (e.g.\  automatically promoted from \double) or
explicitly declared to be of type \expr\ {\em must} be initialized appropriately.
In most cases, this is not a problem since a default \expr\ constructor is 
defined.  \expr\ objects which are dynamically allocated 
using {\tt malloc()} will not be initialized properly.  You should 
use the {\tt new} operator of \cpp\ instead.\\

\begin{progb} {
\> \tt double *pe, *pf;\\
\> \\
\> \tt // The following is incorrect at Levels 2 and 3:\\
\> \tt pe = (double *)malloc(sizeof(double));\\
\> \tt cout << *pe << endl;\\
\> \tt // prints: Segmentation fault\\
\> \tt // because the object *pe was not initialized properly \\
\> \\
\> \tt // This is the correct way:\\
\> \tt pf = new double();\\
\> \tt cout << *pf << endl;\\
\> \tt // prints "0" (the default value)
}\end{progb}

\item
The system's built-in printf and scanf functions cannot be used to 
output/input the \expr\ objects directly. You need to use \cpp\ stream I/O
instead.

\begin{progb}{
\> \tt double e = sqrt(double(2)); \\
\> \tt cout << e << endl; // this outputs 1.4142... depending on current \\
\>\>\>\tt         // output precision and default precision for evaluation \\
\> \tt cin >> e;      // reads from the standard input. 
}\end{progb}

Since we can construct \expr\ objects from strings, you can use
\texttt{scanf} to read a string value which is then assigned to
the \expr\ object.
Unfortunately, current implementation does not support the
use of \texttt{printf}.

\begin{progb}{
\> \tt char s[255]; \\
\> \tt scanf("\%s", s); \\
\> \tt double e = s; 
}\end{progb}

\item
Variables of type \int\ or \float\ are never
promoted to \expr\ objects. For example,

\begin{progb}{
\> \tt // set Level 3\\
\> \tt int i = 2;\\
\> \tt double d = sqrt(i); \\
\> \tt double dd = sqrt(2); 
}\end{progb}
\noindent
The two {\tt sqrt} operations here actually refer to the 
standard {\tt C} function defined in math.h, and not
our exact {\tt sqrt} found in the \expr\ class. 
Hence {\tt d} and {\tt dd} both hold only a 
fixed approximation to $\sqrt{2}$.  The exact value can not be recovered.
Here is a fix:

\begin{progb} {
\> \tt // set Level 3\\
\> \tt int i = 2;\\
\> \tt double e = i; \ \ \ \ \ \ \ \ \ \ \ \ \ // promote i to an \expr\ object\\
\> \tt double d = sqrt(e); \ \ \ \ \ \ \ // the exact sqrt() is called.\\
\> \tt double dd = sqrt(Expr(2)); // the exact sqrt() is called.
}\end{progb}

Users may work around this problem by defining the following macro~:

\begin{progb} {
\> \tt \#define sqrt(x) sqrt(Expr(x))
}\end{progb}

\core\ does not define this since it may be risky for some programs.

\item
In Level 3, constant literals (e.g., 1.3) or constant arithmetic expressions
(e.g., 1/3) are not promoted.   Hence they may not
give exact values.  This can cause some surprises:
 
\begin{center}
\begin{tabular}{c}
\begin{progb} {
\> \tt double a = 1.0/3; \ \ // the value of a is an approximation to 1/3\\
\> \tt double b = 1.3; \ \ \ \ // the value of b is also approximate\\
\> \tt // To input the exact value of 1/3, do this instead: \\
\> \tt double c = BigRat(1, 3); // sure way to get exact value of 1/3\\
\> \tt double d = "1/3"; \ \ // sure way to get exact value of 1/3\\
\> \tt double e = "1.3"; \ \ // the global \definput\ should be\\
\>\>\tt \ \ \ \ \ \ \ \ \ \ \ \ \ \ \ \ // $+\infty$ in order for e to be exact.
}\end{progb}
\end{tabular}
	Program 6
\end{center}

\item
Note that since all the \double\ and \lng\ variables would be 
promoted to \expr\ class during the \candcpp\ preprocessing, 
certain \candcpp\ semantics does not work in
Level 3 anymore. For example, a typical \candcpp\ idiom is the following:

\begin{progb} {
\> \tt double e;\\
\> \tt if (e) \{ \ldots\ \}
}\end{progb}
\noindent
The usual semantics of this code says that 
if the value of {\tt e} is not zero, then do \ldots .
Since {\tt e} is now an \expr\ object, you should write instead:

%% it does not make sense to test whether it is
%% zero or not. What you really want to test is whether the value represented 
%% by e is non-zero.\\

\begin{progb} {
\> \tt double e;\\
\> \tt if (e != 0) \{ \ldots\ \}
}\end{progb}

\ignore{
We can view this as another example of
violating the fundamental rule about treating
all numbers as exact -- make sure
the number API you use is befitting ``real numbers''.
}

\item
Use of standard mathematical functions can be handled in
two ways.  Note that such functions are typically those found
in \texttt{math.h}.   Among these functions, only \texttt{sqrt()}
is fully supported in Level 3.  The other functions such as
\texttt{sin()}, 
\texttt{cos()}, 
\texttt{exp()}, 
\texttt{log()}, etc, are not available in Level 3.
If your program uses these functions, you can invoke
the functions of \texttt{math.h} by explicitly
converting any Level 3 number to machine double as argument:

\begin{progb} {
\> \tt double e = sqrt(2);  \\
\> \tt double s = sin(e.doubleValue());   // this invokes sin() in math.h
}\end{progb}

Thus we invoke the \texttt{Expr::doubleValue()} method
to get a machine double value before calling the sine function.
All our number types have an equivalent \texttt{doubleValue()}
methods.  The second way to call these mathematical functions
is to use our hypergeometric package which computes
these functions to any desired absolute precision.
These functions are found under \progsdir/\texttt{hypergeom}.


\end{enumerate}

\section{Using CORE and CGAL}
  \label{using-with-cgal}

The \cgal\ library (Computational Geometry Algorithm Library, 
\texttt{www.cgal.org}) provides a rich set of geometric primitives
and algorithms.  
These primitives and algorithms are all parametrized
(templated) to receive a variety of number types.  In particular,
you can use \core\ number types.  
We recommend using Level 4, and that you
directly plug \texttt{CORE::Expr} as template parameter of any CGAL kernel~:

\begin{progb} {
\> \tt \#include <CGAL/Cartesian.h> // From CGAL\\
\> \tt \#include "CGAL\_Expr.h"      // From CORE\\
\> \\
\> \tt typedef CGAL::Cartesian<CORE::Expr>  Kernel;\\
\> \tt ...
}\end{progb}

\noindent
\core\ provides some additional functions in the 
file \texttt{inc/CGAL\_Expr.h}, which
are required by \cgal, so you should include this file in your program.
This file sets the {\tt Level} to 4 and 
includes \texttt{CORE.h}.  Some example programs may
be found under \examplesdir\texttt{/cgal}.
\corelib\ is also distributed under \cgal\ by {\tt Geometry Factory},
the company that distributes commercial licenses for \cgal\ components.

\section{Efficiency Issues}

A Level 3 \core\ program can be much less efficient than the
corresponding Level 1 program.  This applies to efficiency
in terms of time as well as space.  First,
Level 3 arithmetic and comparison can be arbitrarily 
slower than the corresponding Level 1 operations.
This is often caused by root bounds which may be far from optimal.
Second,
\expr\ objects can be arbitrarily larger in size than
the corresponding machine number representation.

This inefficiency is partly inherent,
the overhead of achieving robustness in your code.  
But sometimes, part of this overhead not inherent, but caused
by the way your code is structured.
Like any library, \corelib\ gives you the freedom to 
write arbitrarily inefficient programs.
In programming languages too, you can also write inefficient code,
except that modern optimizing compilers can detect
common patterns of suboptimal code and automatically fix it.
This is one of our research goals for \corelib. 
But until now, automatic optimization has not been our primary focus.
This forces users to exercise more care.
The following are hints and tools in \corelib\ that
may speed up your code.  Level 3 is assumed in the following.

\paragraph{Ways to Speed Up your code.}

\begin{itemize}
\item
Share subexpressions.
This requires developing an awareness of how expressions are
built up, and their dependency structure.
Thus, in comparing $E_1=\sqrt{x}+\sqrt{y}$ with
$E_2=\sqrt{x + y + 2\sqrt{xy}}$ (they are equal for all $x, y$),
you could share the $\sqrt{x}$ and $\sqrt{y}$ in $E_1$ and $E_2$
as follows:

\begin{progb} {
\> \tt double x, y, sqrtx, sqrty; \\
\> \tt cin >> x;  cin >> y;		// read from std input\\
\> \tt sqrtx = sqrt(x);  sqrty = sqrt(y);\\
\> \tt double E1 = sqrtx + sqrty; \\
\> \tt double E2 = sqrt(x + y + 2*sqrtx * sqrt y); \\
\> \tt if (E1 == E2) cout << "CORRECT!" << endl; \\
\> \tt else cout << "ERROR!" << endl;
}\end{progb}

See \texttt{prog12.cpp} in in \corepath\texttt{/progs/tutorial} some
timings for shared and non-shared versions of this example.

\item
Avoid divisions in your expressions if possible.
Note that input numbers that are not integers 
are rational numbers, and they implicitly invoke
division.  But a special class of rational numbers,
the $k$-ary numbers (typically, $k=2$ or $10$) can be
quite effectively handled using some new techniques
based on the so-called ``$k$-ary root bounds''.
\core\ Version 1.5 has implemented such binary root bounds ($k=2$).
\item
Be aware that a high algebraic degree can be expensive.
But it is just as important to realize that high algebraic
degree by itself is not automatically a problem.
You can add a hundred positive square roots of integers,
and this algebraic number may have degree up to $2^100$:
\begin{verbatim}
	Expr s = 0;
	for (int i=1; i<=100; i++) s += sqrt(Expr(i));
\end{verbatim}
You can easily evaluate this number \texttt{s}, for example.
But you will encounter trouble when trying to 
compare to such numbers that happen to be identical.

\item
Sometimes, use of expressions are unnecessary.
For instance, consider\footnote{
We are grateful for this example from Professor Siu-Weng Cheng.  
In his example,
the points $p_i$ are 3-dimensional points on the surface
of a unit sphere, which has been computed.  So the coordinates
of $p_i$ are complex expressions involving square roots.
Since we are sure to have a true equality comparison in this
application, the resulting code was
extremely slow using the known root bounds, circa year 2000.
} a piece of code
which iterates over members of a circular list of points until
it reaches the starting member.
If the circular list is $(p_1, p_2,\ldots, p_n)$, and you start
at some point $q=p_i$ in this list, you would normally
think nothing of checking if $q = p_j$ to check if the
entire list has been traversed.  But this can be very 
inefficient.  In this case, you should simply compare
indexes (check if $i=j$), as this does not involve expressions.
\item
Avoid unbounded depth expressions.
A well-known formula \cite{orourke:bk}
for the signed area of a simple polygon $P$ is to add up the signed areas
of each triangle in any triangulation of $P$.  If $P$ has $n$ points,
you would end up with an expression of depth $\Omega(n)$.
If $n$ is large, you will end up with stack overflow.  Even if
there is no overflow, comparing 
this expression to zero (to get its sign) may be too slow.
This example arises in an actual\footnote{
% SHOULD CITE HIS PAPER!
We are grateful to Professor Martin Held for this example.
} software called FIST (``Fast Industrial-Strength Triangulation').
FIST is routinely tested on a database of over 5000 test polygons,
some of which have up to 32,000 points.  FIST uses the above formula
for signed area, and this is too slow
for Core Library (version 1.5) on 32,000 points (but
16,000 points can be handled).
Although it is possible to avoid computing the signed area
when the input polygon is simple, the signed area heuristic is critical
for achieving the ``industrial strength'' properties of FIST.
That is, the signed area approach allows
FIST to produce reasonable triangulations even for ``dirty data''
(such as non-simple polygons).   One solution in this case is to
compute the signed area approximately.
In fact, machine precision approximation is sufficient here.
If more precision is needed, we can use Level 2 numbers (\BF).

\item
Sometimes, \BF\ can be an adequate substitute for expressions.  
For instance, if
you are computing very high precision approximations
of a number, or maintaining isolation intervals of a number
you should use \BF\ for this purpose.   For examples,
see our implementation of Sturm sequences in the distribution.
Here are some useful tips.
First, it is important to know that \BF\ carries an error
bound, and you should probably set this error to zero in such
applications.  Otherwise, this error propagates and
you loose more precision than you might think.  For this purpose,
the following methods are useful to know:
If $x$ is a \BF\ variable, you can find out the error bits
in $x$ by calling $x.{\tt err()}$.  You can test if the error
in $x$ is $0$ by calling $x$.{\tt isExact()}, and if you want
to set this error to zero, call $x$.{\tt makeExact()}.
Sometimes, you need upper or lower bounds on the interval
represented by an inexact \BF\ value.  In this case,
you can call \texttt{makeCeilExact()} or \texttt{makeFloorExact()}.

On the other hand, the inverse method
$x$.{\tt makeInexact()} will set the error to 1.
This is useful for preventing garbage digits from being printed.
For more information, see the file \texttt{BF\_output.cpp} in
the tutorial directory.

Such error bits are positively harmful for self-correcting
algorithms such as Newton iteration -- they may even prevent
Newton iteration from converging.
Note that the expression $x/2$ may not return an exact \BF\ value
even if $x$ is exact.  If you want exact result, you must
call the method $x.{\tt div2()}$.  This is useful when you use
the \BF\ values for interval refinement.

To get your initial \BF\ value, one often
computes an expression $e$ and then
call $e$.{\tt getBigFloat()} to get its current \BF\ approximation.

See Appendix A.1.7 for more information.

\item
A huge rational expression can always be replaced
by an expression with just one node.  This reduction is 
always carried out if a global Boolean variable called
\texttt{rationalReduceFlag} is set to true.
You can set this flag 
by calling \texttt{setRationalReduceFlag(bool)},
which returns the previous value of the flag.
For instance, when this flag is true, the time to run ``make test''
for CORE 1.5 programs takes 1 minute 46.4 seconds;
when this flag is false, the corresponding time is 35.9 seconds.
Hence we do not automatically turn on this flag.  On the other hand,
setting this flag to true can convert an infeasible computation
into a feasible one.  Prior to CORE 1.6, the pentagon
test (\texttt{progs/pentagon/pentagon.cpp}) is an infeasible
computation unless we use the escape precision mechanism (see below).
But with this flag set to true, this test is instantaneous.
Another more significant example is Martin Held's FIST program
-- without this flag, his program will not run with 3D data sets,
crashing because of stack size problems.  

\ignore{
 We implement above by introducing a variable ratFlag in each node.
 We have:  ratFlag < 0 means irrational,
 ratFlag=0 means uninitialized,
 and ratFlag > 0 means rational.
 Currently, ratFlag is an upper bound on the size of the expression, i.e.
 ratFlag(node)=ratFlag(children)+1.
 This would be useful when we introduce transcendentals...
}

\end{itemize}

\paragraph{The Problem of Inexact Inputs.}
A fundamental assumption in EGC is that inputs
are exact.  Even when the application does not care for exactness,
we must treat the input as ``nominally exact''.  This assumption
may fail for some Level 1 programs.  Handling such conversions
will depend on the application, so it is best to illustrate this.
An example is the FIST software above.   Although the basic FIST software
is for triangulating a 2-dimensional polygon, it is also able
to triangulate a 3-dimensional polygon $P$:
FIST will first find the normal of $P$ and then
project $P$ along this normal to the $xy$-plane.  
The problem is thus reduced
to triangulating a 2-dimensional polygon.  In practice, 
the vertices of $P$ are unlikely to lie in a plane.  Hence, the
``normal'' of $P$ is undefined.  Instead, FIST computes the average normal for
each triple of successive vertices of $P$.  
% For instance, when $P$ is
% a pentagon, this results in a normal whose coordinates are expressions
% with over $100$ nodes; the projected coordinates of $P$ involve much
% expressions with up to $119$ nodes.  
The algorithm will further sort the coordinates
of the projected points in order to remove duplicate points.  This
sorting turns out to be extremely expensive in the presence of
duplicates (since in this case, the root bounds of the huge expressions
determine the actual complexity). 
In fact, FIST could not handle this data under CORE 1.5.
It turns out that if we apply the simple
heuristic (within CORE) of reducing all rational expressions
into a single node, the above data could be processed by FIST without
any change in their code, albeit very slowly.  
This heuristic is available from CORE 1.6 onwards.

As an industrial strength software,
FIST must introduce such heuristics to handle inexact data.
To greatly improve the speed of FIST it probably best to change
the logic of its code.  For instance, we suggest converting
the average normal into an approximate floating point representation.
If we further want to preserve the principle that ``exact inputs should produce
exact solutions'', then we can further make FIST to first check that $P$ is
actually planar.  If so, the normal can be determined from
any three non-collinear vertices and hence completely avoid large expressions.
Otherwise, it can use the approximate floating point representation.

\paragraph{Precision Escape Mechanism.}
It is useful to have an escape mechanism to intervene
when a program does not return because of high precision.
This is controlled by the following
two global variables with default values:

\begin{progb} {
\> \tt extLong EscapePrec  = CORE\_INFTY; \\
\> \tt long    EscapePrecFlag = 0; \\
}\end{progb}

When {\tt EscapePrec  = CORE\_INFTY}, the escape mechanism
is not in effect.  But when {\tt EscapePrec} has a finite value 
like $10,000$, then we evaluate the sign of a number, we will not
evaluate its value to an absolute precision that is
more than past $10,000$ bits.  Instead, the {\tt EscapePrecFlag} will
be set to a negative number and we will
\dt{assume} that the sign is really zero. 
Users can check the value of this flag.
This mechanism is applied only in the addition and
subtraction nodes of an expression.
An example of this usage is found \corepath{\tt /progs/nestedSqrt}.

When this mechanism is invoked, the result is no longer guaranteed.
In practice, there is a high likelihood that the \dt{assumed} zero
is really a zero.  That is because root bounds are likely to
be overly pessimistic.

\paragraph{Floating Point Filter.}
It is well-established by recent research
that floating point filters are extremely
effective in avoiding costly big number computations.
We implemented the floating point
filter of Burnikel, Funke and Schirra (BFS).
Note that our implementation, to achieve portability, does
not depend on the IEEE floating point exceptions mechanism.
This filter can be turned off or turned on (the default)
by calling the function \texttt{setFpFilterFlag(bool)}.

\paragraph{Progressive and Non-progressive Evaluation.}
   Users can dynamically toggle a flag to instruct the system to turn off
   progressive evaluation by calling
   {\tt setIncrementalEvalFlag(false)}.  This feature may 
   speed up comparisons that are likely to have
   equality outcomes.
   In applications such as theorem proving (see \cite{tyl:zero-test:00}),
   this may be the case.
However, it does not automatically
lead to better performance even when the comparison result is an equality.
The reason is that when we request a certain number of bits
of precision, the system return a higher precision than necessary.
Hence progressive evaluation may be able to achieve the
desired root bound even though a lower precision is requested,
while going straight to the root bound may cause significant
overshoot in precision.
   To turn back to progressive evaluation, call the
   same function with a {\tt true} argument.

\section{\corelib\ Extensions}

We plan to provide useful \corelib\ extensions
(\corex\ for short).
In the current distribution, we included two simple
\corex's, for linear algebra and for geometry, which the
user may extend to suit their needs.  
The  header files for these \corex's are found in the files
{\tt linearAlgebra.h}, {\tt geometry2d.h} and {\tt geometry3d.h}
under the \includedir.  To use any of these \corex's, just insert
the appropriate include statements: e.g.,
\begin{center}
	{\tt \#include "CORE/linearAlgebra.h"}  
\end{center}
Note that {\tt geometry3d.h} and {\tt geometry2d.h} already
includes {\tt linearAlgebra.h}.
The source for the extensions are found under \extdir.

The {\tt linearAlgebra} extension defines two classes: 
{\tt Matrix} for general $m \times n$ matrices,
and {\tt Vector} for general $n$-dimension vectors. 
They support basic matrix and vector operations. 
Gaussian elimination with pivoting is implemented here.
{\tt Geometry3d} defines classes such as 3-dimensional
{\tt Point}, {\tt Line} and {\tt Plane} based on the linear algebra
API, while {\tt geometry2d} defines the analogous
2-dimensional objects.

The makefile at the top level automatically
builds three versions of the \corex\ libraries,
named {\tt libcorex++\_level1.a},
{\tt libcorex++\_level2.a} and {\tt libcorex++\_level3.a}. 
If you use the \corex\ classes in your own program,
it is important to link
with the correct library depending on the accuracy
level you choose for your program.  
See examples under \examplesdir\ which
use both versions of the \corex\ library
(in particular, \examplesdir {\tt /geom2d},
\examplesdir {\tt /geom3d}, and \examplesdir {\tt /determinant}).

\section{Miscellany}

\paragraph{Invalid Inputs.}
Core will detect the construction of invalid inputs:
this include NaN or Infinity for machine floats and doubles,
divide by zero and square root of negative numbers.
The normal behaviour is to print an error message and abort.
But you can set the \texttt{AbortFlag} to false if you
do not want to automatically abort.  In this case, you
can check if the \texttt{InvalidFlag} is negative.
It is your responsibility to reset this \texttt{InvalidFlag}
to a non-negative value.

\paragraph{Debugging.}
You can output the innards of an expression by calling
the method \texttt{Expr::dump()}.  But these may not be comprehensible
except for the experts.
You can print error or warning messages by calling \texttt{core\_error()},
and these messages will be written into a file \texttt{Core\_diagnostics}.

\paragraph{Variant Libraries.}
It is often useful to store variants of the library simultaneously.
For instance, besides the normal library {\tt libcore++.a},
we may want to use
a variant library called {\tt libcore++Debug.a} which has
information for the debugger, or some other variant
which implement some new techniques.
In our {\tt Makefile}s, we use a variable {\tt VAR} whose
value {\tt \$\{VAR\}} is normally set to the empty string.
This variable is defined in the file \corepath{\tt /make.config}.
To produce a debugging version of the library, set this
variable to the string "{\tt Debug}".  Then,
in the \srcdir, type ``{\tt make}'' to automatically
create the library {\tt libcore++\$\{VAR\}.a} which will
be put in \libdir\ as usual.  Finally, to use the
debugging version of the library, call \gpp\ with
the library option {\tt -lcore++\$\{VAR\}} instead of {\tt -lcore++}.

\section{Bugs and Future Work}
\label{sec-problems}

The ability of a single program to access all of four
accuracy levels has not been fully implemented.
Currently, support for Levels 2 and 4 is fairly basic.
Absolute precision in Level 3 is not optimized: the system
always determines the signs of expressions, which is sometimes
unnecessary.  It would be useful to extend
\expr\ to (1) allow interval values in leaves and
(2) construct real roots of polynomials whose  
coefficients are expressions (i.e., diamond operator).
% and approximate values are always in \BF.

We started to address many of the I/O issues raised in
previous versions: number formats (scientific, positional, rational),
the ability to read and write from files, especially in hex.
But there is room for improvement.
Bivariate polynomials and plane algebraic curves 
were introduced in Version 1.7, and are expected to be
developed over the next versions.

\ignore{
Input and Output can be further improved.
(1) There are 2 number formats for inputs and outputs: scientific and
positional.  One should allow a third format,
for rational number input and output (e.g., $1/3$ should be available
instead of $0.333\ldots$).
(2) One should be able to read and write outputs in hex notation,
thus avoid the expense of converting between binary and
decimal representations.
This is useful for computation with transcendental functions
which can be sped up by table lookup of high precision constants.
(3) It would be nice to be able to force ``exact'' output when
possible (essentially, allow output precision to be $\infty$).
Irrational values will be printed with some
form of indication (e.g. trailing dots to indicate
irrational value: $1.41421356...$) but rational values will
be printed in rational form.
This means our system will need to detect when a value is
actually rational. 
(4) We should be able to input and output \expr\ and \real\ objects
using some character string format.
}

Future plans include
better floating point filters,
% (including avoiding any root bound computation until these fail),
special determinant subsystem,
optimized Level 2,
optimized implementation of absolute precision bounds,
complex algebraic numbers,
incremental arithmetic,
more efficient root bounds,
expression optimization (including common subexpression detection),
expression compilation for partial evaluation,
transcendental functions,
templated versions of Linear algebra and geometry extensions,
graphical facilities,
enriched \corex's.

We would like to hear your suggestions, experience and bug reports
at 
	\mbox{\tt exact@cs.nyu.edu}.\\

%%%%%%%%%%%%%%%%%%%%%%%%%%%%%%%%%%%%%%%%%%%%%%%%%%%%%%%%%%%%%%
% APPENDIX
%%%%%%%%%%%%%%%%%%%%%%%%%%%%%%%%%%%%%%%%%%%%%%%%%%%%%%%%%%%%%%
% appendix.tex
%   -- include file for tutorial.tex

\newpage
\appendix
\section{APPENDIX: \core\ Classes Reference}
\label{classes}

There are three main classes in the \core\ package: \expr, \real\ and \BF.
The \expr\ class is built upon the other two classes and provides 
the basic functionalities of Level 3 accuracy.  Although users do not have 
to directly access the \real\ and \BF\ classes, they are useful for 
understanding the behavior of the \corelib.
Advanced users may want to program directly with these classes.
Here is a brief summary of these classes.

\begin{description}
\item[\real] is a ``heterogeneous'' number system\footnote{
In contrast, most number systems has a ``homogeneous'' representation.
The \real\ class ought to provide automatic conversions
among subtypes, but this capability is not
currently implemented.
} that currently incorporates the following six subtypes: 
\int, \lng, \double, \Int, \Rat, and \BF.
The first three are standard machine types while
the latter three are big number types.
Since Version 1.6, \Int\ and \Rat\ are wrapper classes
for gmp's mpz and mpq.

\item[\expr] is the most important class of the library.
It provides the mechanism to support Level 3 accuracy.
An \expr\ object has an approximate value as well as a precision.
Users can freely set the precision of the \expr\ object,
and its approximate value will be automatically adjusted to
satisfy the precision.  Currently, the approximate value is a \BF.

\item[\BF] is an arbitrary precision floating point number
representation that we built on top of \Int.  It is used by
our library to represent approximate values.  
A \BF\ is represented by the triple $\langle m, \varepsilon, e \rangle$
where $m$ is the mantissa of type \Int,
$\varepsilon$ is the error bound and $e$ is the exponent. 
It represents the interval $(m\pm \varepsilon)B^e$
where $B=2^{14}$.  These intervals
are automatically maintained when performing arithmetic with \BF's.
\end{description}

Besides these three classes, the user should know about the
class \extlong\ of ``extended longs''.
This is a wrapper around the primitive \lng\ type with
the special values of \posInfty, \negInfty, and \NaN.
For convenience, {\tt CORE\_INFTY} is defined to be \posInfty.
By using these special values,
extended longs can handle overflows as well
as undefined operations (divide by zero) in a graceful way.
This class is extensively used in specifying root bounds and precision.

The rest of this appendix is a reference for the classes
\BF, \real\ and \expr.


\subsection{The Class \BF}
A \BF\ number $x$ is given as a triple $\left< m, err, exp \right> $ 
where the {\em mantissa} is $m$, the {\em error-bound} is
$err \in \left\{ 0, 1, \ldots, B-1 \right\}$ and 
$exp$ is the {\em exponent}. Here the {\em base} $B$ is equal to $2^{14}$.

The ``number'' $x$ really represents the interval
	\begin{equation}
	\label{eq:bigfloat}
	\left[ \left( m - err \right) B^{exp}, \left( m + err \right) B^{exp} \right]
	\end{equation}
We say that a real number $X$ {\em belongs} to $x$ if
$X$ is contained in this interval.
In our implementation of \BF, $m$ is \Int, $err$ is
{\tt unsigned long}, and $exp$ is {\tt long} for efficiency. 
You can obtain these components of a \BF\ by calling
the member functions \texttt{m()}, \texttt{err()} and \texttt{exp()}.
Since Version 1.4, \Int\ is gmp's Big Integer.

If $err = 0$ then we say the \BF\ $x$ is {\em error-free}. When we
perform the operations $ +, -, *, / $ and $ \sqrt{\phantom{\cdot}} $
on \BF\ numbers, the error-bound is automatically propagated subject in the
following sense:
{\em
if $X$ belongs to \BF\ $x$ and $Y$ belongs
to \BF\ $y$, and we compute \BF\ $z=x\circ y$ (where $\circ\in\{+,-,\times,\div\}$)
then $X\circ Y$ belongs to $z$.
}  A similar condition holds for the unary operations.
In other words, the error-bound in the result $z$ must be ``large enough''.

There is leeway in the choice of the error-bound in $z$.
Basically, our algorithms try to minimize the error-bound in $z$ subject to
efficiency and algorithmic simplicity.
This usually means that the error-bound in $z$ is within a small
constant factor of the optimum error-bound
(see Koji's thesis \cite{ouchi:thesis} for full details).
But this may be impossible
if both $x$ and $y$ are error-free: in this case, the optimum error-bound
is $0$ and yet the result $z$ may not be representable
exactly as a \BF.  This is the case for
the operations of $\div$ and $\sqrt{\cdot}$.
In this case, our algorithm ensures that
the error in $z$ is within some default precision (the value of
global variable {\tt defAbsPrec}).
This is discussed under the class \real\ below.  

A practical consideration in our design of the class \BF\ is
that we insist that the error-bound $err$
is at most $B$.  To achieve this, we may
have to truncate the number of significant
bits in the mantissa $m$ in (\ref{eq:bigfloat}) and modify the exponent $exp$
appropriately.

\subsubsection{Class Constructors for \BF}
\label{sec-bf-cons}
\begin{progb}{
\> \tt 	 BigFloat();\\
\> \tt 	 BigFloat(int);\\
\> \tt 	 BigFloat(long);\\
\> \tt 	 BigFloat(double);\\
\> \tt 	 BigFloat(const BigInt\& M, unsigned long err = 0, long exp = 0);\\
\> \tt   BigFloat(const \BF \&);\\
\> \tt 	 BigFloat(const char *);\\
\> \tt 	 BigFloat(const std::string\&);\\
\> \tt 	 BigFloat(const \Rat\ \&,  const extLong\& r = defRelPrec,\\
\> \> \> \tt            const extLong\& a = defAbsPrec);\\
\> \tt 	 BigFloat(const \expr\ \&, const extLong\& r = defRelPrec,\\
\> \> \> \tt            const extLong\& a = defAbsPrec);
}\end{progb}

The default constructor declares an instance with a value
zero. The instances of \BF\ can also be constructed from \int,
\lng, \float, \double, \Int\ and \str.  
The last two constructors needs some clarification:
(a)
The constructor from strings is controlled by the global parameter \defBFinput,
and ensures that the \BF\ value constructor differs from the
string value by an absolute value of at most $10^{-\defBFinput}$.
(b)
The constructors {\tt \BF(\Rat\ R, r, a)} 
and {\tt \BF(\expr\ e, r, a)} constructs a
\BF\ that approximates the rational {\tt R} and expression {\tt e}
to the composite precision {\tt [r, a]}. 

\nopagebreak
\begin{progb}{
\> \tt BigInt I(5); \\
\> \tt BigFloat B(I); \\
\> \tt BigFloat bf1("0.023");\\
\> \tt BigFloat bf2("1234.32423e-5");\\
\> \tt BigRat R(1, 3);\\
\> \tt BigFloat br(R, 200, CORE\_INFTY);
}\end{progb}

\subsubsection{Assignment}

\begin{progb} {
\> \tt	BigFloat\& operator=(const BigFloat\&);\\
\> // arithmetic and assignment operators\\
\> \tt	BigFloat\& operator+=(const BigFloat\&);\\
\> \tt	BigFloat\& operator-=(const BigFloat\&);\\
\> \tt	BigFloat\& operator*=(const BigFloat\&);\\
\> \tt	BigFloat\& operator/=(const BigFloat\&);
}\end{progb}

\subsubsection{Arithmetic Operations}
\begin{progb} {
\> \tt	BigFloat operator+(const BigFloat\&, const BigFloat\&); \\
\> \tt	BigFloat operator-(const BigFloat\&, const BigFloat\&); \\
\> \tt	BigFloat operator*(const BigFloat\&, const BigFloat\&); \\
\> \tt	BigFloat operator/(const BigFloat\&, const BigFloat\&); \\
\> \tt	BigFloat sqrt(const BigFloat\&);
}\end{progb}

\subsubsection{Comparison}
\begin{progb} {
\> \tt	bool operator==(const BigFloat\&, const BigFloat\&); \\
\> \tt	bool operator!=(const BigFloat\&, const BigFloat\&); \\
\> \tt	bool operator< (const BigFloat\&, const BigFloat\&); \\
\> \tt	bool operator<=(const BigFloat\&, const BigFloat\&); \\
\> \tt	bool operator> (const BigFloat\&, const BigFloat\&); \\
\> \tt	bool operator>=(const BigFloat\&, const BigFloat\&);
}\end{progb}

\subsubsection{Approximations}
\label{sec-bf-approx}
\begin{progb} {
\> \tt  void approx(const BigInt\& I, const extLong\& r, const extLong\& a);\\
\> \tt  void approx(const BigFloat\& B, const extLong\& r, const extLong\& a);\\
\> \tt  void approx(const BigRat\& R, const extLong\& r, const extLong\& a);
}\end{progb}

Another important source of \BF\ numbers is via
the approximation of \Int, \BF\ and \Rat\ numbers.  We provide
the member functions {\tt approx} which take such a number, 
and precision bounds $r$ and $a$ and assign to the \BF\ a value
that approximates the input number to the specified composite
precision bounds:

\nopagebreak
\begin{progb}{
\> \tt BigRat R(1,3); // declares {\tt R} to have value 1/3. \\
\\ 
\> \tt BigFloat B; \\
\> \tt B.approx(R,16,16); \\
\> \tt // now {\tt B} contains an approximation of 1/3 to precision [16,16].\\
}\end{progb}

\subsubsection{Conversion Functions}
\label{sec-bigfloat-cast}
\begin{progb} {
\> \tt 	 double doubleValue() const; \>\>\>\>\>\>\> // convert to machine built-in double\\
\> \tt 	 float  floatValue()  const; \>\>\>\>\>\>\> // convert to machine built-in float\\
\> \tt 	 long   longValue()   const; \>\>\>\>\>\>\> // convert to machine built-in long\\
\> \tt 	 int    intValue()    const; \>\>\>\>\>\>\> // convert to machine built-in int\\
\> \tt 	 \Int\  BigIntValue() const; \>\>\>\>\>\>\> // convert to a BigInt number\\
\> \tt 	 \Rat\  BigRatValue() const; \>\>\>\>\>\>\> // convert to a BigRat number\\
}\end{progb}

The semantics of these expressions are mostly self-explanatory.
For conversion of \Int, we simply use truncation of the mantissa of the \BF.
Users must exercise caution in using these conversions. 
Overflow or underflow errors occur silently during the conversion. 
It is the user's responsibility to detect such conditions.

\subsubsection{Algebraic Operations}

\begin{progb}{
\>\tt	bool isDivisible(const \BF\ \&a, const \BF\ \&b); \\
\>\tt	\BF\ div\_exact(const \BF\ \&a, const \BF\ \&b); \\
\>\tt	\BF\ gcd(const \BF\ \&a, const \BF\ \&b); 
}\end{progb}

All algebraic operations are only defined for exact \BF\ values.
The methods \texttt{isDivisible},
\texttt{div\_exact} and
\texttt{gcd} are global functions. 
The definition of divisibility is not obvious: every floating point value
can be uniquely written in the form $m 2^e$ where $m$ is an odd integer
and $e$ is an integer.  We say $m2^e$ divides $m'2^{e'}$ if
$m$ divides $m'$ and $|e|\le |e'|$ and $ee'\ge 0$.
Once this concept is defined, the meaning of these algebraic
operations are standard.

\subsubsection{I/O}
\label{sec-BF-out}

\begin{progb} {
\> \tt 	 ostream\& operator<<(ostream\&, const BigFloat\&);\\
\> \tt 	 istream\& operator>>(istream\&, BigFloat\&);
}\end{progb}

Stream I/O operators are defined for \BF. Integer values can be read in exactly.
Fractional values are read in correctly within an absolute error
of $10^{-\defBFinput}$
where \defBFinput\ is a global parameter settable by users.
Note that\\
\defBFinput\ cannot be $\infty$.

Outputs utilize the precision parameter $p$ associated with output
streams. The parameter $p$ is interpreted to be the number of digits printed:
in scientific format, this is the number of significant digits
but in positional format, leading zeros are counted.  E.g.,
$0.123$ and $0.001$ both have $p=4$.
When outputting, the error bits in a \BF\ representation
are first truncated. The output is in one of two formats: 
{\em positional} or {\em scientific}.
In scientific
notation, the length of the mantissa can at most be $p$. The extra
digits are rounded (to the closest possible output value). 
We choose scientific
notation if at least one of the following conditions is true:
\begin{enumerate}
\item The scientific notation flag is on. This flag is associated with 
output streams in C++ and can be set using standard I/O manipulator.
\item The absolute value of the number is smaller than $10^{-p+1}$.
\item The absolute value of the number is bigger or equal to
$10^{p-\delta}$ where $\delta=0$ or $1$ depending on whether
the number is a whole number or not.
\end{enumerate}

Note that we may actually print out less than $p$ digits
if the \BF\ value does not have that many digits of precision.
If a \BF\ $x$ is not error-free,
the output is a decimal number whose value
approximates the value of $x$ correctly to the last digit.
That means that the last significant digit $m_\ell$ really lies in the 
range $d_\ell\pm 1$, where $d_\ell$ is the last digit of output.

\ignore{
Next, the output routine will print out at most 
{\tt defPrtDgt} digits using a rounding to the nearest digit rule. 
If the absolute
value of the exponent is equal to
or larger than {\tt defPrtDgt}, the output 
strings use the scientific notation of the form 
$$m_1.m_2m_3\cdots m_\ell ~\mbox{\bf e} \pm e_1e_2\cdots e_m$$
where the $m$'s and $e$'s are in decimal notation and $\mbox{\bf e}$
is a literal character indicating the start of the exponent. Otherwise,
the usual decimal representation would be adopted. 
Note that we may actually print out less than {\tt defPrtDgt}
many digits.

This output represents decimal floating point number whose value
approximates the value of $x$ correctly to the last digit.
That means that the last significant
digit $m_\ell$ really lies in the range $m_\ell\pm 1$.
The number of significant digits in this output is $\ell$, and
it is controlled by the global variable \defprt.  The value
of \defprt\ is defaulted to 10, but the user can change this
at run time.  The value $\ell$ is 
equal to the minimum of \defprt\ and the number of significant
digits in the internal representation of $x$.
}

It is interesting to see the interplay between ostreams' {\tt precision} 
$p$ and the composite precision {\tt [defAbsPrec, defRelPrec]}.
Keep in mind that {\tt defAbsPrec} and {\tt defRelPrec} refer to binary bits.

\begin{center}
\begin{tabular}{c}
\begin{progb} {
\> \tt double q = BigRat(1, 3); \\
\> \tt setDefaultAbsPrecision(67); // about 20 digits \\
\> \tt cout << "q = " << setprecision(10) << q << ", in 10 digits" << endl;\\
\> \tt   // output: q = 0.33333333, in 10 digits\\
\> \tt cout.precision(30); // or use setDefaultOutputDigits(30, cout), \\
\> \tt                     // default to output 30 digits.\\
\> \tt cout << "q = " << q << endl;\\
\> \tt  // output: q = 0.33333333333333333333, in positional notation.\\
}\end{progb}
\end{tabular}
	Program 7
\end{center}

\ignore{ A LOW LEVEL WAY TO DO THE ABOVE:

double q = BigRat(1, 3);
setDefaultAbsPrecision(67); // about 20 digits
cout << "q = " << setprecision(10) << q << ", in 10 digits" << endl;
  // output: q = 0.33333333, in 10 digits
cout.precision(30); // default to output 30 digits.
cout.setf(ios::scientific, ios::floatfield); // use scientific notation
cout << "q = " << q << ", in scientific notation." << endl;
  // output: q = 3.3333333333333333333e-1, in scientific notation.
cout.setf(0, ios::floatfield); // reset the format to default.
cout << "q = " << q << ", in positional notation." << endl;
  // output: q = 0.33333333333333333333, in positional notation.

} %ignore


\ignore{ OLD!
\nopagebreak
\begin{progb}{
\> \tt setDefaultOutputDigits(4);\\
\> // sets the number of output digits to 4.\\
\> \tt cout << B;\\
\> // prints the value 0.333.\\
\> \tt setDefaultOutputDigits(6) ;\\
\> \tt cout << B;\\
\> // prints the value 0.33333.\\
\> \tt setDefaultOutputDigits(20);\\
\> \tt cout << B;\\
\> // prints the value 0.3333333320915699005.\\ 
\> // the precision is not high enough to get all printed digits right.\\
\> \tt setDefaultPrecision(67, \posInfty);\\
\> // set adequate precision for 20 correct significant decimal digits.\\
\> \tt cout << B;\\
\> // prints the value 0.3333333333333333333
}\end{progb}
} %ignore

It is programmers' responsibility to
set the composite precisions high enough to have all requested digits printed correctly.

\subsubsection{Miscellaneous}

To get the sign of the mantissa in a \BF, use
	$$\texttt{int BigFloat::sign()}$$
which returns one of the values $-1, 0, +1$.  
Since there may be error in a \BF, this may not
be taken as the sign of the \BF\ unless you also verify
that the following predicate is false:
	$$\texttt{bool BigFloat::isZeroIn()}.$$
This predicate is true iff $0$ lies in the interval
$[mantissa \pm err]$.  

Another useful function is
	$$\texttt{BigFloat::isExact()}$$
which returns a true Boolean value
if the error component of the \BF\ is zero.
We can set this error component to zero by calling
	$$\texttt{BigFloat::makeExact()}.$$
There are two variants, \texttt{makeCeilExact()} and
\texttt{makeFloorExact()}.  Why would one do this?
There are applications where the error bound is not needed.  
An example is when you implement a Newton iterator for roots.
The algorithm is self-correcting, so the error bound is
not necessary.  But after an inexact operation (e.g., division),
the error bound is nonzero.  If you do not set this to zero, our
automatic significance arithmetic algorithms
may start to truncate the mantissa in order to keep the error bound
from growing.  This may prevent Newton from converging.

The most significant bit (MSB) of
any real number $x$ is basically $\lg|x|$
(log to base 2).  When $x$ is an exact \BF, this value is an integer.
In general, $x$ has an error, and it represents an interval
of the form $[x-\varepsilon, x+\varepsilon]$.  We provide two functions,
\texttt{BigFloat::uMSB()} and \texttt{BigFloat::lMSB()},
each returning an extended long.  Assuming that $0< x-\varepsilon$, 
the following inequalities must hold
	$$ \texttt{BigFloat::lMSB()} \le \lg |x-\varepsilon|
		\le \lg | x+\varepsilon| \le \texttt{BigFloat::uMSB()}.$$
When $x+\varepsilon < 0$, we modify the inequalities approproriately.
When \texttt{x.isExact()} is true, the
inequalities become equalities, and
you can simply call the function \texttt{x.MSB()}.
Here is an application: suppose you have computed
a \BF\ value $x$ approximates $\sqrt{2}$.   To see
how close $x$ is to $\sqrt{2}$, you can compute
	$$\texttt{extLong p = (x*x - 2).uMSB() }.$$
E.g., to guarantee that $x- \sqrt{2}< 2^{-100}$, it is enough
to make sure that this \textit{p} is less than $-97$.
An alternative approach is to compare $x*x -2$ to $2^{-97}$.
You can obtain the value $2^{-97}$ by calling another helper function
	$$\texttt{BigFloat::exp2(int n)}$$
which returns a \BF\ whose value is $2^n$.

A \BF\ is really a wrapper about a \texttt{BigFloatRep} object.
Sometime, you may like to get at
this ``rep'', using the member function \texttt{getRep()}.

\subsection{The Class \real}
\label{sec-real-intro}
The class \real\ provides a uniform interface to the six subtypes of numbers:
\int, \lng, \double, \Int, \Rat, and \BF. 
There is a natural type coercion among these types as one would
expect. It is as follows:

\begin{center}
$ \int \prec \lng \prec \double \prec \BF \prec \Rat $ , \\
$ \int \prec \lng \prec \Int \prec \Rat $ .
\end{center}

The \BF\ in this coercion is assumed to be error-free. 
To use the class \real, a program simply includes the file
{\tt Real.h}.

\nopagebreak
\begin{progb}{
\> \tt \#include "CORE/Real.h"
}\end{progb}

\subsubsection{Class Constructors for \real}
% \samepage

\begin{progb}{
\> \tt  Real();\\
\> \tt  Real(int);\\
\> \tt  Real(long);\\
\> \tt  Real(double);\\
\> \tt  Real(const BigInt\&);\\
\> \tt  Real(const BigRat\&);\\
\> \tt  Real(const BigFloat\&);\\
\> \tt  Real(const Real\&);\\
\> \tt  Real(const char *str, const extLong\& prec = defInputDigits); \\
\> \tt  Real(const std::string\& str, const extLong\& prec = defInputDigits);
}\end{progb}

The default constructor declares an instance with a default 
\real\ with the value zero.
Consistent with the \cpp\ language, an instance can be
initialized to be any subtype of \real: \int, \lng, \double, \Int,
\Rat, and \BF. 

In the last constructor from string, the conversion is exact in three cases:
(i) when the value is integral (i.e., the string has only digits);
(ii) rational (i.e., the string contains one `/' character and digits otherwise);
(iii) when {\tt prec} = \coreInfty.
Otherwise, it will convert to a \BF\ number within absolute precision
$10^{-\mathit{prec}}$.

\bigskip

\subsubsection{Assignment}
% \samepage
\begin{progb} {
\> \tt Real\& operator=(const Real\&);\\
\\
\>  // arithmetic and assignment operators\\
\> \tt	Real\& operator+=(const Real\&); \\
\> \tt	Real\& operator-=(const Real\&); \\
\> \tt	Real\& operator*=(const Real\&); \\
\> \tt	Real\& operator/=(const Real\&); \\
\\
\>  // post- and pre- increment and decrement operators\\
\> \tt	Real operator++(); \\
\> \tt	Real operator++(int); \\
\> \tt	Real operator--(); \\
\> \tt	Real operator--(int);
}\end{progb}

Users can assign values of type \int, \lng, \double, \Int, \Rat, and
\BF\ to any instance of \real.

\nopagebreak
\begin{progb}{
\> \tt Real X; \\
\\
\> \tt X = 2; // assigns the machine \int\ 2 to {\tt X}.
\\
\> \tt X = BigInt(4294967295); // assigns the \Int\ 4294967295 to {\tt X}.
\\
\> \tt X = BigRat(1, 3); // assigns the \Rat\ 1/3 to {\tt X}.
}\end{progb}

\bigskip
\subsubsection{Arithmetic Operations}
\begin{progb} {
\> \tt   Real operator-() const; \\
\> \tt 	 Real operator+(const Real\&) const; \\
\> \tt 	 Real operator-(const Real\&) const; \\
\> \tt 	 Real operator*(const Real\&) const; \\
\> \tt 	 Real operator/(const Real\&) const; \\
\> \tt 	 Real sqrt(const Real\&);
}\end{progb}

The class \real\ supports binary operators {\tt +, -, *, /} and the
unary operators {\tt -}, and {\tt sqrt()} with standard
operator precedence.

% \samepage
The rule for the binary operators $bin\_op\in\{+,-,\times\}$ is as follows:
let $ typ_X $ and $ typ_Y $ be
the underlying types of \real\ {\tt X} and {\tt Y}, respectively.
Then the type of \mbox{{\tt X} $ bin\_op $ {\tt Y}}
would be $MGU=max \{ typ_1, typ_2 \} $ where the order $ \prec $ is defined
as in the type coercion rules in Section~\ref{sec-real-intro}.  
For instance, consider Program 8.  Although {\tt X} and {\tt Y}
are of type {\tt RealInt}, their sum {\tt Real Z} is 
of type {\tt RealBigInt} since the value of {\tt Z} cannot be represented
in {\tt RealInt}.

\nopagebreak
\begin{center}
\begin{tabular}{c}
\begin{progb}{
\> \tt Real X, Y, Z; \\
\> \tt unsigned int  x, y, z; \\
\> \tt int  xx, yy, zz; \\
\\
\> \tt X = 1; x = 1; xx = 1;\\
\> \tt Y =  4294967295; // $ 2^{32}-1 $\\
\> \tt y =  4294967295; // $ 2^{32}-1 $ \\
\> \tt yy = 2147483647; // $ 2^{31}-1 $ \\
\> \tt Z = X + Y;  z = x + y;  zz = xx + yy; \\
\> \tt cout << "Z = " << Z << endl; \ // prints: Z = 4294967296 (correct!)\\
\> \tt cout << "z= " << z << endl;  \ // prints: z = 0 (overflow)\\
\> \tt cout << "zz= " << zz << endl;  // prints: z = -2147483648 (overflow)\\
}\end{progb}
\end{tabular}
	Program 8
\end{center}

\paragraph{Square Root.}
The result of {\tt sqrt()} is always a \BF.
There are two cases: in case the original input has an error $err>0$,
then the result of the {\tt sqrt()} operation has
error-bound at most $16\sqrt{err}$ , see \cite{ouchi:thesis}.
If $err=0$, then the absolute error of the result is
at most $2^{-a}$ where $a$=\defabs.

\nopagebreak
\begin{progb}{
\> \tt Real X = 2.0; \\
\\
\> \tt cout << setprecision(11) << sqrt(X) << endl; \\
\> // prints: 1.414213562
}\end{progb}

\paragraph{Division.}
The type $typ_Z$ of {\tt Z = X / Y} is either \Rat\ or \BF. 
If both $typ_X$ and $typ_Y$ are not \float, \double\ or \BF, then
$typ_Z$ is \Rat; otherwise, it is \BF.

If the output type is \Rat, the output is exact.
For output type of \BF, the error bound in {\tt Z} is determined as follows.
Inputs of type \float\ or \double\ are considered to be error-free,
so only \BF\ can have positive error.
If the error-bounds in {\tt X} or {\tt Y} are positive, then
the relative error in {\tt Z} is at most $12\max\{relErr_X,relErr_Y\}$
where $relErr_X,relErr_Y$ are the relative errors in {\tt X}
and {\tt Y}, respectively.
If both {\tt X} and {\tt Y} are error-free then the
relative error in {\tt Z} is at most \defrel.


\subsubsection{Comparison}
% \samepage
\begin{progb} {
\> \tt	bool operator==(const Real\&) const;\\
\> \tt	bool operator!=(const Real\&) const;\\
\> \tt	bool operator< (const Real\&) const;\\
\> \tt	bool operator<=(const Real\&) const;\\
\> \tt	bool operator> (const Real\&) const;\\
\> \tt	bool operator>=(const Real\&) const;
}\end{progb}

\subsubsection{Real I/O}
\label{sec-real-out}
% \samepage

\begin{progb} {
\> \tt  istream\& operator>>(istream\&, Real);\\
\> \tt  ostream\& operator<<(ostream\&, const Real\&);
}\end{progb}
The input string is parsed and if the value is integral, it is
read in exactly. Otherwise, it is approximated with absolute
precision \definput\ (in decimal). Note that if
\definput\ is $\infty$, the input is read in exactly
as a big rational number. 

To output the value of an instance of \real, we can use the standard
\cpp\ output stream operator {\tt <<}.
Output is in decimal representation.
There are two kinds of decimal outputs:
for \int, \lng, \Int\ and \Rat\ subtypes, this is the exact
value of \real.  But for \double\ and \BF\ subtypes, we use
the decimal floating point notation described under
\BF\ output. 

\nopagebreak
\begin{center}
\begin{tabular}{c}
\begin{progb}{
\> \tt BigRat R(1, 3); \\
\> \tt BigFloat B(R); \\
\> \tt BigInt   I = 1234567890;\\
\> \tt cout.precision(8); // set output precision to 8\\
\\
\> \tt Real Q = R; \\
\> \tt Real X = B; \\
\> \tt Real Z = I; \\
\\
\> \tt cout << R << endl;   // prints: \tt 1/3 \\
\\
\> \tt cout << Q << endl;   // prints: \tt 0.3333333 \\
\\
\> \tt cout << X << endl;   // prints: \tt 0.3333333\\
\\
\> \tt cout << Z << endl;   // prints: \tt .12345679e+10
}\end{progb}
\end{tabular}\\
	Program 9
\end{center}

Other related functions are

\begin{progb}{
\> \tt int BigInt::fromString(const char* s, int base = 0);
\\
\> \tt std::string BigInt::toString(int base = 10);
\\
}\end{progb}

If {\tt base = 0}, then a prefix {\tt p} in string {\tt s}
tself determines the base: {\tt p = 0b} means binary,
{\tt p = 0} means octal,
{\tt p = 0x} means hexadecimal,
{\tt p} is the empty string means decimal.

\subsubsection{Approximation}
\begin{progb} {
\> \tt Real approx(const extLong\& r, const extLong\& a) const;
}\end{progb}

Force the evaluation of the approximate value to the composite 
precision $[r, a]$. The returned value is always a {\tt RealBigFloat}
value.
\ignore{
Actually it is built upon the \BF\ functions described in
Section~\ref{sec-bf-approx}.
}

\subsubsection{Miscellaneous}
\begin{progb} {
\> // get the sign of a Real value\\
\> \tt  int Real::sign() const;\\
\> \tt  int sign(const Real\&);
}\end{progb}

\subsection{The Class \expr}

To use the class \expr, a program simply includes the file
{\tt Expr.h}. The file {\tt Real.h} is automatically included 
with {\tt Expr.h}.

\nopagebreak
\begin{progb}{
\> \tt \#include "CORE/Expr.h"
}\end{progb}

For most users,
the ideal way to use our library is to have the user
access only the class \expr\ indirectly by setting the accuracy level
to 3 so that \double\ and \lng\ will be prompted to \expr.

An instance of the class \expr\ $ E $ is formally a triple
$$ E = ( T, P, A )$$
where $T$ is an {\em expression tree},
$P$ a composite precision,
and $A$ is some real number or $\uparrow$ (undefined value).
The internal nodes of $T$
are labeled with one of those operators
	\begin{equation}\label{eq:op}
	 +, -, \times, \div, \sqrt{\cdot},
	\end{equation}
and the leaves of $T$ are labeled by \real\ values or is $\uparrow$.
$P=[r,a]$ is a pair of \extlong, with $ r $ non-negative.
If all the leaves of $T$ are labeled by \real\ values, then
there is a real number $V$ that is the value of the expression $T$;
otherwise, if at least one leaf of $T$ is labeled by $\uparrow$, then $V=\uparrow$.
Finally, the value $A$ satisfies the relation
	$$A\simeq V [r,a].$$
This is interpreted to mean either $V=A=\uparrow$ or
$A$ approximates $V$ to precision $P$.
In the current implementation, leaves must hold exact values.
Moreover, the value $A$ is always a \BF.
The nodes of expression trees are instances of the
class \exprep.  More precisely,
each instance of \expr\ has a member \rep\ that points
to an instance of \exprep.  Each instance of \exprep\ is allocated on
the heap
and has a type, which is either one of the operations
in (\ref{eq:op}) or type ``constant''.
Depending on its type, each instance of \exprep\ has zero, one or two pointers
to other \exprep.  For instance, a constant \exprep,
a $\sqrt{\cdot}$-\exprep\ and a +-\exprep\ has
zero, one and two pointers, respectively.
The collection of all \exprep s together
with their pointers constitute a directed acyclic graph (DAG).
Every node $N$ of this DAG defines an expression tree $E(N)$ in the
natural way.
Unlike \cite{ouchi:thesis}, assignment to \expr\ has the standard semantics.
As an example, after the assignment $e = f \odot g$, 
$\val(e) = \val(f) \odot \val(g)$ and $\val(e)$ does not change
until some other assignment to $e$. In particular, subsequent assignments
to $f$ and $g$ do not affect $\val(e)$.

\bigskip

\subsubsection{Class Constructors for \expr}

% \samepage

\begin{progb}{
\> \tt  Expr(); \\
\> \tt  Expr(int); \\
\> \tt  Expr(long); \\
\> \tt  Expr(unsigned int); \\
\> \tt  Expr(unsigned long); \\
\> \tt  Expr(float); \\
\> \tt  Expr(double); \\
\> \tt  Expr(const BigInt \&); \\
\> \tt  Expr(const BigFloat \&); \\
\> \tt  Expr(const BigRat \&); \\
\> \tt  Expr(const char *s, const extLong\& prec=defInputDigits); \\
\> \tt  Expr(const std::string \&s, const extLong\& prec=defInputDigits); \\
\> \tt  Expr(const Real \&); \\
\> \tt  Expr(const Expr \&); // copy constructor \\
\> \tt  template<class NT> \\
\>\> \tt  Expr(const Polynomial<NT>\& p, int n=0);\\
\>\>\>	// this specifies the n-th smallest real root. \\
\> \tt  template<class NT> \\
\>\> \tt  Expr(const Polynomial<NT>\& p, const BFInterval\& I);\\
\>\>\>	// this specifies the unique real root in interval I.
}\end{progb}

The default constructor of \expr\ constructs a constant
\expr\ object with the value zero.
When a constructor is called with some \real\ value, then a 
parameter which contains the specified \real\ value is declared.

In \corelib\ 1.6, we introduced a new constant \expr\ object which is
constructed from a polynomial (see Appendix A.4 for 
the \Poly\ class).  The value is a real root of
this polynomial, and so we need an argument to indicate
a unique root.  This can be an closed interval $I$
comprising a pair of \BF's, or an integer $n$.  
Collectively, both $n$ and $I$ are called {\em root indicators}.
The interval $I$ must be {\em isolating} meaning
that it contains a unique real root of the polynomial.
If $n\ge 1$, then we specify the $n$th
{\em smallest} real root (so $n=1$ is the smallest real root).
If $n\le -1$, this refers to the $(-n)$-th {\em largest} real root.

For convenience, we also provide three global functions to help the user
construct such \expr\ node:

\begin{progb}{
\> \tt  template<class NT> \\
\> \> \tt  rootOf(const Polynomial<NT>\& p, int n=0); \\
\> \tt  template<class NT> \\
\> \> \tt  rootOf(const Polynomial<NT>\& p, const BFInterval\& I); \\
\> \tt  template<class NT> \\
\> \> \tt  radical(const NT\& k, int m); // the m-th root of k\\
}\end{progb}

So for a polynomial {\tt P}, both 
{\tt Expr e(P, i)} and {\tt Expr e = rootOf(P, i)} are
equivalent (where {\tt i} is a root indicator).


\ignore{
\nopagebreak
A variable is declared if a constructor is called with an algebraic
expression which involves operators
{\tt +}, {\tt -}, {\tt *}, {\tt /} or {\tt sqrt()},
instances of \expr\ (can be either parameters or variables), and
\real\ values. However, the expressions must be dags (i.e., cycles
are not allowed).
}

\bigskip	
\subsubsection{Assignments}
\begin{progb} {
\> \tt   Expr\& operator=(const Expr\&); \\ 
\\
\> \tt 	 Expr\& operator+=(const Expr\&); \\
\> \tt 	 Expr\& operator-=(const Expr\&); \\
\> \tt 	 Expr\& operator*=(const Expr\&); \\
\> \tt 	 Expr\& operator/=(const Expr\&); \\
\\
\> \tt 	 Expr\& operator++(); \\
\> \tt 	 Expr   operator++(int); \\
\> \tt 	 Expr\& operator--(); \\
\> \tt 	 Expr   operator--(int);
}\end{progb}



%%\subsubsection{Assignments}
%%
%%Two distinct instances of the class \expr\ may share their \rep\
%%s. Assignments to instances of \expr\ must therefore be handled
%%carefully.
%%
%%Depending on what is on the right hand side of the assignment, there
%%are three distinct meanings for assignments.
%%
% %%\samepage
%%If the RHS is a \real\ value, then the assignment is called {\em value
%%assigning}; the LHS becomes a parameter.
%%
%%\nopagebreak
%%\begin{progb}{
%%\> \tt Expr e, f; \\
%%\> \tt Real X; \\
%%\\
%%\> \tt e = X; \\
%%\> \tt f = 5; \\
%%\> // value assigning; \\
%%\> // {\tt e} and {\tt f} are now parameters.
%%}\end{progb}
%%
% %%\samepage
%%If the RHS is an algebraic expression, then the assignment is called
%%{\em constructing}; the LHS becomes a variable.
%%
%%\nopagebreak
%%\begin{progb}{
%%\> \tt Expr e, f, g; \\
%%\\
%%\> \tt e = f * g; \\
%%\> // {\tt e} is now a variable.
%%}\end{progb}
%%
%%\bigskip
%%

\subsubsection{Arithmetic Operations}
% \samepage
\begin{progb} {

\> \tt Expr operator-() const; //unary minus \\
\> \tt Expr operator+(const Expr\&, const Expr\&); //addition \\
\> \tt Expr operator-(const Expr\&, const Expr\&); //subtraction \\
\> \tt Expr operator*(const Expr\&, const Expr\&); //multiplication \\
\> \tt Expr operator/(const Expr\&, const Expr\&); //division \\
\> \tt Expr sqrt(const Expr\&); // square root \\
\> \tt Expr abs(const Expr\&); // absolute value \\
\> \tt Expr fabs(const Expr\&); // same as abs()  \\
\> \tt Expr pow(const Expr\&, unsigned long); // power \\
\> \tt Expr power(const Expr\&, unsigned long); // power \\
}\end{progb}

% WHAT ABOUT the 

% \samepage
For the convenience and efficiency,
integer powers can be constructed by
applying the function {\tt power()}.
	
\nopagebreak
\begin{progb}{
\> \tt Expr e = 3 * power(B, 5);\\
\> \>     // alternative for "Expr e = 3 * B*B*B*B*B.\\
}\end{progb}

\subsubsection{Comparisons}
% \samepage

\begin{progb} {
\> \tt  bool operator==(const Expr\&, const Expr\&); \\
\> \tt  bool operator!=(const Expr\&, const Expr\&); \\
\> \tt  bool operator< (const Expr\&, const Expr\&); \\
\> \tt  bool operator<=(const Expr\&, const Expr\&); \\
\> \tt  bool operator> (const Expr\&, const Expr\&); \\
\> \tt  bool operator>=(const Expr\&, const Expr\&);
}\end{progb}

The standard \cpp\ comparison operators {\tt <}, {\tt >}, {\tt <=},
{\tt >=}, {\tt ==}, and {\tt != } perform ``exact comparison''. When
{\tt A < B} is tested, {\tt A} and {\tt B } are evaluated to
sufficient precision so that the decision is made correctly. Because
of root bounds, such comparisons always terminate. The
returned value is a non-negative integer, where 0 means ``false'' while
non-0 means ``true''.

%  // An example of O'Rourke:
\nopagebreak
\begin{center}
\begin{tabular}{c}
\begin{progb}{
\> \tt Expr e[2];\\
\> \tt Expr f[2];\\
\> \tt e[0] = 10.0; e[1] = 11.0;\\
\> \tt f[0] = 5.0;  f[1] = 18.0;\\
\> \tt Expr ee = sqrt(e[0])+sqrt(e[1]);\\
\> \tt Expr ff = sqrt(f[0])+sqrt(f[1]);\\
\> \tt if (ee>ff) cout << "sr(10)+sr(11) > sr(5)+sr(18)" << endl;\\
\> \tt     else cout << "sr(10)+sr(11) <= sr(5)+sr(18)" << endl;\\
\> // prints: $sr(10)+sr(11) > sr(5)+sr(18)$\\
}\end{progb}
\end{tabular}\\
	Program 10
\end{center}

\bigskip

\subsubsection{Expr I/O}

% % \samepage

\begin{progb} {
\> \tt  ostream\& operator<<(ostream\&, const Expr\&); \\
\> \tt  istream\& operator>>(istream\&, Expr \&);
}\end{progb}

The input will construct a {\tt ConstRep} with a {\tt Real} value
read in from the input stream. The input routine for {\tt Real} is
discussed in Section~\ref{sec-real-out}.

The standard \cpp\ operator {\tt <<} outputs the stored approximate 
value which is always a \BF\ number. If there is no
approximate value available, it will force an evaluation to the
default precisions.
It prints as many digits of significance as is currently known as correct 
(up to the output precision specified). See Section~\ref{sec-BF-out} for 
examples.

\subsubsection{Approximation}
\begin{progb} {
\> \tt Real approx(const extLong\& r = defRelPrec, const extLong\& a = defAbsPrec);
}\end{progb}


{\tt A.approx(r, a)} evaluates {\tt A} and returns its approximate value
to precision {\tt [r, a]}. If no argument is passed, then {\tt A} is
evaluated to the default global precision [\defrel, \defabs]. If the
required precision is already satisfied by the current approximation, 
the function just returns the current approximate value. 

An expression is not evaluated until the evaluation is 
requested explicitly (e.g., by {\tt approx()}) or implicitly 
(e.g.\ by some I/O operations).

\nopagebreak
\begin{progb}{
\> \tt Expr e; \\
\> \tt Real X; \\
\> \tt unsigned r; int a; \\
\\
\> \tt X = e.approx(r, a); \\
\> // {\tt e} is evaluated to precision at least \tt [r, a] \\
\> // and this value is given to {\tt X};
}\end{progb}

The following helper functions allow you to get at the
current approximate value in an \expr:

\nopagebreak
\begin{progb}{
\> \tt Expr e; \\
\> \tt $\vdots$ \\
\> \tt e.sign(); // returns the exact sign of e (note that e.getSign() is deprecated,\\
\> \>\>\> as "sign()" is the uniform interface for all the number classes\\
\> \tt e.BigFloatValue(); // returns the current BigFloat approximation \\
\> \tt e.getMantissa(); // returns the mantissa of current BigFloat \\
\> \tt e.getExponent(); // returns the exponent of current BigFloat \\
}\end{progb}

\subsubsection{Conversion Functions}
\label{sec-expr-cast}
\begin{progb} {
\> \tt 	 double doubleValue() const; \>\>\>\>\>\>\> // convert to machine built-in double\\
\> \tt 	 float  floatValue()  const; \>\>\>\>\>\>\> // convert to machine built-in float\\
\> \tt 	 long   longValue()   const; \>\>\>\>\>\>\> // convert to machine built-in long\\
\> \tt 	 int    intValue()    const; \>\>\>\>\>\>\> // convert to machine built-in int\\
\> \tt 	 \Int\  BigIntValue() const; \>\>\>\>\>\>\> // convert to a BigInt number\\
\> \tt 	 \Rat\  BigRatValue() const; \>\>\>\>\>\>\> // convert to a BigRat number\\
\> \tt 	 \BF\  BigFloatValue() const; \>\>\>\>\>\>\> // convert to a BigFloat number\\
}\end{progb}

The semantics of these operations are clear except for converting into
\Rat\ or \BF.  For \BF, we use the current approximate  value of the
expression. For \Rat, we use the same \BF\ value converted into a rational number.
Note that users must exercise caution in using these conversions. 
Overflow or underflow errors occur silently during the conversion. 
It is the user's responsibility to detect such conditions.
Nevertheless, they are useful for converting 
existing \candcpp\ programs.  E.g.,
these operators can be applied on the {\tt printf()} arguments. See 
Section~\ref{sec-convert} for details.

\subsection{Filters and Root Bounds}
\label{sec-expr-filters}
The expression class has
an elaborate mechanism for computing root bounds,
and a floating point filter.
Our filters is based on the so-called BFS Filter \cite{bfs:exact-cascaded:01}.
Our root bounds are a combination of several techniques
(BFMSS Bound, Measure bound and conjugate bound).
In fact, the BFMSS bound is the so-called k-ary version
\cite{pion-yap:kary:03}.  For more details on these topics, see
\cite{li-pion-yap:progress:04}.


\subsection{The Template Class \Poly}

Class \Poly\ is a template class, which can be instantiated with the 
number type {\tt NT} of polynomial coefficients.   
We support {\tt NT} chosen from \int\, \Int\, \BF\, \Rat\ and \expr.

Since Version 1.6, the Class \Poly\ is incorporated into \corelib.
In particular, the file \texttt{CORE.h} or \texttt{Expr.h} automatically
include the files \texttt{poly/Poly.h} and \texttt{poly/Poly.tcc}.
The following constructors are available for this class:

\begin{progb}{
\> \tt Polynomial();    // the Zero Polynomial \\
\> \tt Polynomial(int n);   // the Unit Polynomial of nominal deg $n\ge 0$ \\
\> \tt Polynomial(int n, NT* coef); // coef is the array of coefficients\\
\> \tt Polynomial(const VecNT \&); // VecNT is a vector of coefficients\\
\> \tt Polynomial(int n, const char * s[]);\\
\> \tt Polynomial(const Polynomial \&); \\
\> \tt Polynomial(const string \& s, char myX='x' );\\
\> \tt Polynomial(const char * s, char myX='x' ); 
}\end{progb}

The last two constructors takes a string {\tt s}.
They are convenient and intuitive to use, and 
works best for up to moderate size polynomials.
For instance, The user can construct a polynomial
by calling \mbox{\tt Polynomial p("3x\^2 + 4*x + 5")} using
the default variable name {\tt x}.  If you use some
other variable name such as {\tt Z}, then you can use
the second argument to specify this.  E.g.,
\mbox{\tt Polynomial p("3Z \^2 + 4*Z  + 5", 'Z')}.
The syntax for a valid input string {\tt s} given by the following
BNF grammar:

\begin{verbatim}
[poly] -> [term] | [term] '+/-' [poly] \\
                    | '-' [term] | '-' [term] '+/-' [poly] \\
[term] -> [basic term] | [basic term] [term] | [basic term]*[term]\\
[basic term] -> [number] | 'x'
                    | [basic term] '^' [number] | '(' [poly] ')'
\end{verbatim}

The recursiveness in these rules meant that an input string such as
{\tt s = "(2x - 1)\^12 (x\^2 - 2x + 3)"} is valid.
See \progsdir/\texttt{poly/parsePoly.cpp} for examples. 

When specifying an array {\tt coef} of coefficients,
the coefficient of the power product $x^i$
is taken from {\tt coeff[i]}.  So the constant term is {\tt coeff[0]}.
If we want to reverse this ordering (and treat
{\tt coeff[0]} as the leading coefficient),
we can first use the above constructor, and then
reverse the polynomial (the reverse method is listed below).

An example of how to use these constructors are shown below:

\begin{progb}{
\> \tt typedef BigInt NT; \\
\> \tt typedef Polynomial<NT> PolyNT;  // convenient typedef \\
\> \tt PolyNT P1;     // Zero Polynomial \\
\> \tt PolyNT P2(10); // Unit Polynomial of degree 10\\
\> \tt NT coeffs[] = \{1, 2, 3\}; \\
\> \tt PolyNT P3(1, coeffs); // P3(x) = 1 + 2x + 3x**2\\
\> \tt const char* s[] = \{"123456789", "0", "-1"\}; \\
\> \tt PolyNT P4(1, s) // P4(x) = 123456789 - x**2; \\
\> \tt PolyNT P5("u\^2(u + 234)\^2 - 23(u + 2)*(u+1)", 'u');
}\end{progb}

You can also input these coefficients as strings (this
is useful when the coefficients are so large that
they may overflow a machine integer).

\subsubsection{Assignments}

\begin{progb} {
\> \tt   Polynomial\& operator=(const Polynomial\&); \\ 
\\
\> \tt 	 Polynomial\& operator+=(const Polynomial\&); \\
\> \tt 	 Polynomial\& operator-=(const Polynomial\&); \\
\> \tt 	 Polynomial\& operator*=(const Polynomial\&); 
}\end{progb}

\subsubsection{Arithmetric Operations}

\begin{progb} {
\> \tt   Polynomial\& operator-(); \\ 
\\
\> \tt 	 Polynomial\& operator+(const Polynomial\&, const Polynomial\&); \\
\> \tt 	 Polynomial\& operator-(const Polynomial\&, const Polynomial\&); \\
\> \tt 	 Polynomial\& operator*(const Polynomial\&, const Polynomial\&); 
}\end{progb}

\subsubsection{Comparisons}

\begin{progb} {
\> \tt 	 bool operator ==(const Polynomial\&, const Polynomial\&); \\
\> \tt 	 bool operator !=(const Polynomial\&, const Polynomial\&); 
}\end{progb}

\subsubsection{I/O}

\begin{progb} {
\> \tt 	 ostream\& operator<<(ostream\&, const Polynomial\&);\\
\> \tt 	 istream\& operator>>(istream\&, Polynomial\&);
}\end{progb}

\subsubsection{Manipulation and Query Functions}

The following methods are used to manipulate (i.e., modify)
to query polynomials:

\begin{progb} {
\> \tt  int expand(int n);      // Change the nominal degree to n \\
\> \tt  int contract();	        // get rid of leading zeros \\
\> \tt  int getDegree() const;	// nominal degree \\
\> \tt  int getTrueDegree() const;  // true degree \\
\> \tt  const NT\& getLeadCoeff() const; // get TRUE leading coefficient \\
\> \tt  const NT\& getTailCoeff() const; // get last non-zero coefficient\\
\> \tt  NT** getCoeffs() ;		// get all coefficients \\
\> \tt  const NT\& getCoeff(int i) const; // Get coefficient of $x^i$ \\
\> \tt  bool setCoeff(int i, const NT\& cc);i // Makes cc the coefficient\\
\>\>	// of $x^i$; return FALSE if invalid i. \\
\> \tt  void reverse();		// reverse the coefficients \\
\> \tt  Polynomial \& negate(); //Multiply by -1.\\
\> \tt  int makeTailCoeffNonzero(); // Divide (*this) by $x^k$, so that\\
\>\>    // the tail coeff is non-zero. Return k.\\
}\end{progb}

\subsubsection{Algebraic Polynomial Operations}

\begin{progb}{
\>\tt  Polynomial\& differentiate();   // self-differentiation\\
\>\tt  Polynomial\& differentiate(int n); // multi self-differentiation\\
\>\tt  Polynomial\& squareFreePart();  // P/gcd(P,P') \\
\>\tt  Polynomial\& primPart(); // Primitive Part \\
\>\tt  Polynomial pseudoRemainder (const Polynomial\& B, NT\& C);\\
\>\>  // The pseudo quotient of (*this) mod B is returned, but (*this) is\\
\>\>  // transformed into the pseudo remainder.  If argument C is not not\\
\>\>  // null, then C*(*this) = B*pseudo-quotient + pseudo-remainder.\\
\>\tt   Polynomial \& negPseudoRemainder (const Polynomial\& B); \\
\>\>    // Same as the previous one, except negates the remainder.\\
\>\tt  Polynomial reduceStep (Polynomial\& p ); \\
}\end{progb}

All of the above operations are self-modifying.  If this
is undesirable, the user ought to make a copy of the polynomial first.

\subsubsection{Numerical Polynomial Operations}

These operations include evaluation and root bounds:

\begin{progb}{
\> \tt  Expr eval(const Expr\&) const;		// polynomial evaluation \\
\> \tt  BigFloat eval(const BigFloat\&) const;	// polynomial evaluation \\
\> \tt  template <class myNT>  myNT eval(const myNT\&) const;
		// evaluation at an\\
\> \>   //arbitrary number type.\\
\> \tt  BigFloat CauchyUpperBound() const;  // Cauchy Root Upper Bound \\
\> \tt  BigFloat CauchyLowerBound() const;  // Cauchy Root Lower Bound \\
\> \tt  BigFloat sepBound() const;	// separation bound (multiple roots allowed) \\
\> \tt  BigFloat height() const;		// height function\\
\> \tt  BigFloat length() const;		// length function\\
}\end{progb}

Note that the {\tt eval} function here is a generic function:
it allows you to evaluate the polynomial at any number type
{\tt myNT}.  The return type is also {\tt myNT}.  To do this,
we convert each coefficient (which has type {\tt NT})
of the polynomial into type {\tt myNT}.  Then all the operations
of the evaluation is performed within the class {\tt myNT}.
For this to work properly, we therefore require that
{\tt NT}$\le${\tt myNY} (recall that there is a natural
partial ordering among number types).   For instance, if {\tt NT}=\BF,
then {\tt myNT} can be \BF, \Rat\ or \expr.   
In particular, using {\tt myNT} will ensure exact results; but this
may be expensive and in many
situations, \BF\ is the correct choice (e.g., Newton iteration).

\subsubsection{Miscellaneous}
Some methods in \Poly\ depend on the choice of {\tt NT}.
In particular, some methods need to know whether the coefficient
type {\tt NT} supports\footnote{
	We say ``general division'' to distinguish
	this from special kinds of division such as
	division by 2 (this is supported by \BF)
	or exact division (this is supported by \Int).
} ``general'' division.  Hence we require all such
number types to provide a static method
	{\tt NT::hasDivision()}
that returns a boolean value.  Among the supported {\tt NT},
only \Rat\ and \expr\ has general division.


\subsection{The Template Class \Sturm}

% Class \Sturm\ has been incorporated into \corelib\ since 1.6.
This class implements the Sturm sequence associated with a polynomial. 
Starting with Version 1.7 this class can handle \int, \Int, \lng,
\Rat, \BF, and \expr. The most important being \Int, \BF, and \expr,
although the last one can be inefficient for polynomials with large degree.
The constructors are:

\begin{progb}{
\> \tt  Sturm(); // null constructor\\
\> \tt  Sturm(PolyNT pp); // constructor from polynomial\\
\> \tt  Sturm(int n, NT * c);// constructor from an array of coefficients\\
\> \tt  Sturm(const Sturm\& s);  // copy constructor
}\end{progb}

After we have constructed a Sturm object based upon some polynomial, we can 
use the following functions to get more properties as described below.

\subsubsection{Functions in \Sturm\ Class}

\begin{progb}{
\> \tt  int signVariations(const BigFloat\& x, int sx); \\
\>\> // Gets the sign variations of the Sturm sequence at a given point\\
\> \tt  int signVariationsAtPosInfty();\\
\> \tt  int signVariationsAtNegInfty();\\
\> \tt  int numberOfRoots(const BigFloat\& x, const BigFloat\& y);\\
\> \> //Number of roots in the closed interval [x, y] \\
\> \tt  int numberOfRoots();// number of real roots of the polynomial\\
\> \tt  int numberOfRootsAbove(const BigFloat \&x);\\
\> \tt  int numberOfRootsBelow(const BigFloat \&x);\\
\> \tt  void isolateRoots(const BigFloat \&x, const BigFloat \&y,
                    BFVecInterval \&v);\\
\> \> //Isolates all the roots in the interval [x,y] and returns them in v\\
\> \> //a list of intervals\\
\> \tt  void isolateRoots(BFVecInterval \&v); \rm// Isolates all the roots\\
\> \tt  BFInterval isolateRoot(int i); \rm// Isolate the i-th smallest\\
\> \> // root, if $i < 0$ then we get the i-th largest root\\
\> \tt  BFInterval isolateRoot(int i, BigFloat x, BigFloat y);\\
\> \> // Isolate the i-th smallest root in the interval [x,y]\\
\> \tt  BFInterval firstRootAbove(const BigFloat \&e);\\
\> \tt  BFInterval firstRootBelow(const BigFloat \&e);\\
\> \tt  BFInterval mainRoot();//First root above 0\\
\> \tt	BFInterval refine(const BFInterval\& I, int aprec);\\
\> \> // Refine the interval \texttt{I} containing the root using bisection\\
\> \tt  BFInterval refinefirstRootAbove(const BigFloat \&e, int aprec);\\
\> \> //Get an absolute approximation to aprec of the first root above \texttt{e}.\\
\> \> //Achieved using the refine method above.\\
\> \tt  BFInterval refinefirstRootBelow(const BigFloat \&e, int aprec);\\
\> \> // Similar to previous method, except refines the first root below \texttt{e}\\
\> \tt  void refineAllRoots( BFVecInterval \&v, int aprec);\\
\> \> //Refines all the roots to absolute precision aprec (based upon refine)\\
\> \tt
}\end{progb}

A main feature of the Sturm Class is that it provides standard
Newton iteration using which we can converge rapidly to any root of the 
underlying polynomial. The following methods provide the desired
functionality.


\subsubsection{Newtons Method in \Sturm\ Class}

\begin{progb}{

\> \tt BigFloat newtonIterN(long n, const BigFloat\& bf, BigFloat\& del,\\
\> \>                        unsigned long \& err);\\
\> \> // Does n steps of standard Newton's method starting from the initial\\
\> \> // value bf. The return value is the approximation to the root after\\
\> \> // n steps. del is an exact BigFloat which is an upper bound on the  \\
\> \> difference between the n-th and n-1-th approximation, say $del_{n-1}$.\\
\> \>  err is an upper bound $|del - del_{n-1}|$.\\
\> \tt BigFloat newtonIterE(int prec, const BigFloat\& bf, BigFloat\& del);\\
\> \> // Does Newton iteration till $del.uMSB() < -prec$\\
\> \tt BFInterval newtonRefine(const BFInterval I, int aprec);\\
\> \> // Given an isolating interval I for a root x*, will return \\
\> \> //an approximate root x such that $|x-x^*| < 2^{-aprec}$.\\
\> \> //Assumes that the interval end points are known exactly.\\
\> \tt void newtonRefineAllRoots( BFVecInterval \&v, int aprec);\\
\> \> // Refines all the roots of the polynomial to the desired precision\\
\> \> // aprec using newtonRefine above\\
\> \tt   bool smaleBoundTest(const BigFloat\& z); // Implementation of \\
\> \> // Smale's point estimate to determine whether we have reached \\
\> \> // Newton basin. This is an a posteriori criterion unlike the next.\\
\> \tt   BigFloat yapsBound(const Polynomial<NT> \& p);// An apriori bound\\
\> \> // to determine whether we have reached Newton zone.\\

}\end{progb}


\subsection{The Template Class \Curve}

Introduced in Version 1.7, this class allows the user 
to manipulate arbitrary real algebraic curves. 
The \Curve\ class is derived from the \BiPoly\ class,
so we begin by describing the \BiPoly\ class:

\begin{progb}{
\> \tt  BiPoly(); //Constructs the zero bi-poly.\\
\> \tt	BiPoly(int n);// creates a BiPoly with nominal y-degree of n.\\
\> \tt  BiPoly(std::vector<Polynomial<NT> > vp); // From vector of Polynomials\\
\> \tt  BiPoly(Polynomial<NT> p, bool flag=false);\\
\> \>  //if true, it converts polynomial p(x) into p(y)\\
\> \>  //if false, it creates the bivariate polynomial y - p(x)\\
\> \tt  BiPoly(int deg, int *d, NT *C);  //Takes in a list of list of\\
\> \>  // coefficients.	Each cofficient list represents a polynomial in x\\
\> \>  //  deg - ydeg of the bipoly\\
\> \>  //  d[] - array containing the degrees of each coefficient\\
\> \>  //          (i.e., x poly)\\
\> \>  //  C[] - list of coefficients, we use array d to select the \\
\> \>  //        coefficients.\\
\> \tt   BiPoly(const BiPoly<NT>\&); //Copy constructor\\
\> \tt   BiPoly(const string\& s, char myX='x', char myY='y');\\
\> \tt   BiPoly(const char* s, char myX='x', char myY='y');
}\end{progb}


The last two constructors from strings
are similar to the ones for \Poly.  The syntax of valid input string
is determined by a BNF grammar that is identical to the
one for univariate polynomials, except that we now
allow a second variable {\tt 'y'}.


\subsubsection{Assignments}
\begin{progb} {
\> \tt  BiPoly<NT> \& operator=( const BiPoly<NT>\& P); // Self-assignment\\
\> \tt  BiPoly<NT> \& BiPoly<NT>::operator+=( BiPoly<NT>\& P); // Self-addition\\
\> \tt  BiPoly<NT> \& BiPoly<NT>::operator-=( BiPoly<NT>\& P); //Self-subtraction\\
\> \tt BiPoly<NT> \& BiPoly<NT>::operator*=( BiPoly<NT>\& P); //Self-multiplication}\end{progb}

\subsubsection{Comparison and Arithmetic}
\begin{progb}{
\> \tt bool operator==(const BiPoly<NT>\& P, const BiPoly<NT>\& Q);\\
\> \> //Equality operator for BiPoly\\
\> \tt BiPoly<NT> operator+(const BiPoly<NT>\& P, const BiPoly<NT>\& Q);\\
\> \> //Addition operator for BiPoly\\
\> \tt BiPoly<NT> operator-(const BiPoly<NT>\& P, const BiPoly<NT>\& Q);\\
\> \> //Subtraction operator for BiPoly\\
\> \tt BiPoly<NT> operator*(const BiPoly<NT>\& P, const BiPoly<NT>\& Q);\\
\> \> //Multiplication operator for BiPoly
}\end{progb}

\subsubsection{I/O}

\begin{progb}{
\> \tt  void dump(std::ostream \& os, std::string msg = "");\\
\> \tt  void dump(std::string msg="");\\
}\end{progb}

These dump the \BiPoly\ object to a file or standard output as a string.

\subsubsection{Functions}

We have the following methods to manipulate bivariate polynomials.

\begin{progb}{
\> \tt Polynomial<NT> yPolynomial(const NT \& x); // Returns the univariate\\
\> \> //polynomial obtained by evaluating the coeffecients at x.\\
\> \tt  Polynomial<Expr> yExprPolynomial(const Expr \& x);\\
\> \>   // Expr version of yPolynomial.\\
\> \tt  Polynomial<BigFloat> yBFPolynomial(const BigFloat \& x);\\
\> \>   // BF version of yPolynomial\\
\> \tt  Polynomial<NT> xPolynomial(const NT \& y) ;\\
\> \>   //   returns the polynomial (in X) when we substitute Y=y\\
\> \tt  int getYdegree() const; // returns the nominal degree in Y\\
\> \tt  int getXdegree(); // returns the nominal degree in X.\\
\> \tt  int getTrueYdegree();//returns the true Y-degree.\\
\> \tt  Expr eval(Expr x, Expr y);//Evaluate the polynomial at (x,y)\\
\> \tt  int expand(int n);\\
\> \>   // Expands the nominal y-degree to n;\\
\> \>   // Returns n if nominal y-degree is changed to n, else returns -2\\
\> \tt  int contract();\\
\> \>   // contract() gets rid of leading zero polynomials\\
\> \>   // and returns the new (true) y-degree; returns -2 if this is a no-op\\
\> \tt  BiPoly<NT> \& mulXpoly( Polynomial<NT> \& p);\\
\> \>   // Multiply by a polynomial in X\\
\> \tt  BiPoly<NT> \& mulScalar( NT \& c);\\
\> \>   //Multiply by a constant\\
\> \tt  BiPoly<NT> \& mulYpower(int s);\\
\> \>  // mulYpower: Multiply by $Y^i$ (COULD be a divide if i<0)\\
\> \tt  BiPoly<NT> \& divXpoly( Polynomial<NT> \& p);\\
\> \>   // Divide by a polynomial in X.\\
\> \>   // We replace the coeffX[i] by the pseudoQuotient(coeffX[i], P)\\
\> \tt  BiPoly<NT>  pseudoRemainderY (BiPoly<NT> \& Q);\\
\> \>   //Using the standard definition of pseudRemainder operation.\\
\> \>   //	--No optimization!\\
\> \tt  BiPoly<NT> \& differentiateY();  //Partial Differentiation wrt Y\\
\> \tt  BiPoly<NT> \& differentiateX();  //Partial Differentiation wrt X\\
\> \tt  BiPoly<NT> \& differentiateXY(int m, int n);\\
\> \> //m times wrt X and n times wrt Y\\
\> \tt  BiPoly<NT> \& convertXpoly();\\
\> \>   //Represents the bivariate polynomial in (R[X])[Y] as a member\\
\> \>   //of (R[Y])[X].  This is needed to calculate resultants w.r.t. X.\\
\> \tt   bool setCoeff(int i, Polynomial<NT> p);\\
\> \>   //Set the $i$th Coeffecient to the polynomial passed as a parameter\\
\> \tt  void reverse();// reverse the coefficients of the bi-poly\\
\> \tt  Polynomial<NT> replaceYwithX();\\
\> \tt  BiPoly<NT>\& pow(unsigned int n);  //Binary-power operator\\
\> \tt  BiPoly<NT> getbipoly(string s);\\
\> \>   //Returns a Bipoly corresponding to s, which is supposed to\\
\> \>   //contain as place-holders the chars 'x' and 'y'.\\
}\end{progb}

There are other useful friend functions for \BiPoly\ class:

\begin{progb}{
\> \tt bool zeroPinY(BiPoly<NT> \& P);\\
\> \> //checks whether a Bi-polynomial is a zero Polynomial\\
\> \tt BiPoly<NT> gcd( BiPoly<NT>\& P ,BiPoly<NT>\& Q);\\
\> \> //   This gcd is based upon the subresultant PRS to avoid\\
\> \> //   exponential coeffecient growth and gcd computations, both of which \\
\> \> //   are expensive since the coefficients are polynomials\\
\> \tt Polynomial<NT>  resY( BiPoly<NT>\& P ,BiPoly<NT>\& Q);\\
\> \> //      Resultant of Bi-Polys P and Q w.r.t. Y.\\
\> \> //      So the resultant is a polynomial in X\\
\> \tt BiPoly<NT>  resX( BiPoly<NT>\& P ,BiPoly<NT>\& Q);\\
\> \> //      Resultant of Bi-Polys P and Q w.r.t. X.\\
\> \>//      So the resultant is a polynomial in Y\\
\> \>//	We first convert P, Q to polynomials in X. Then \\
\> \>// 	call resY and then turn it back into a polynomial in Y\\
}\end{progb}



We now come to the derived class \Curve.  All the methods
provided for bivariate polynomials are available for curves as well,
but there are two additional functions:

\begin{progb}{
\> \tt   int verticalIntersections(const BigFloat \& x, BFVecInterval \& vI,\\
\>\> \> 			  int aprec=0);\\
\> \>  // The list vecI is passed an isolating intervals for y's such \\
\> \>  // that (x,y) lies on the curve.\\
\> \>  // If aprec is non-zero (!), the intervals have with $<2^{-aprec}$.\\
\> \>  // Returns $-2$ if curve equation does not depend on Y,\\
\> \>  //       -1 if infinitely many roots at x,\\
\> \>  //    	0 if no roots at x,\\
\> \>  //    	1 otherwise\\
\> \tt  int  plot( BigFloat eps=0.1, BigFloat xmin=-1.0,\\
\> \> \>     BigFloat ymin=-1.0, BigFloat xmax=1.0, BigFloat ymax=1.0, int fileNo=1);\\
\> \>  //    Gives the points on the curve at resolution "eps".  Currently,\\
\> \>   //    eps is viewed as delta-x step size.\\
\> \>   //    The display is done in the rectangle [xmin, ymin, xmax, ymax].\\
\> \>   //    The output is written into a file in the format specified\\
\> \>   //    by our drawcurve function (see COREPATH/ext/graphics).\\
\> \>   //    Heuristic: the open polygonal lines end when number of roots\\
\> \>   //    changes.
}\end{progb}

%% END
	% Classes reference
\newpage
\section{APPENDIX: Sample Program}
\label{sec-example-dt}
\label{example}
 
The following is a simple program from O'Rourke's book to 
compute the Delaunay triangulation for $n$ points.
The program tests all triples of points to see
if their interior is empty of other points, and outputs
the number of ``empty'' triples. 
In our adaptation of O'Rourke's program below,
we generate input points that are (exactly) co-circular.
This highly degenerate set of input points is expected
to cause problems at Level 1 accuracy.


{\scriptsize
\begin{verbatim}
----------------------------------------------------------------------
#define CORE_LEVEL 3     // Change "3" to "1" if you want Level 1 accuracy
#include "CORE/CORE.h"
main() {                 // Adapted from O'Rourke's Book
  double x[1000],y[1000],z[1000];/* input points x y,z=x^2+y^2 */
  int    n;                      /* number of input points */
  double xn, yn, zn;             /* outward vector normal to (i,j,k) */
  int    flag;                   /* true if m above (i,j,k) */
  int    F = 0;                  /* # of lower faces */
  // define the rotation angle to generate points
  double sintheta = 5;  sintheta /= 13;
  double costheta = 12; costheta /= 13;

  printf("Please input the number of points on the circle: ");
  scanf("%d", &n);
  x[0] = 65;  y[0] = 0; z[0] = x[0] * x[0] + y[0] * y[0];
  for (int i = 1; i < n; i++ ) {
    x[i] = x[i-1]*costheta - y[i-1]*sintheta; // compute x-coordinate
    y[i] = x[i-1]*sintheta + y[i-1]*costheta; // compute y-coordinate
    z[i] = x[i] * x[i] + y[i] * y[i];         // compute z-coordinate
  }
  for (int i = 0; i < n - 2; i++ )
    for (int j = i + 1; j < n; j++ )
      for (int k = i + 1; k < n; k++ ) 
        if ( j != k ) {
          // For each triple (i,j,k), compute normal to triangle (i,j,k). 
          xn = (y[j]-y[i])*(z[k]-z[i]) - (y[k]-y[i])*(z[j]-z[i]);
          yn = (x[k]-x[i])*(z[j]-z[i]) - (x[j]-x[i])*(z[k]-z[i]);
          zn = (x[j]-x[i])*(y[k]-y[i]) - (x[k]-x[i])*(y[j]-y[i]);
          if ( flag = (zn < 0) ) // Only examine faces on bottom of paraboloid
            for (m = 0; m < n; m++)
              /* For each other point m, check if m is above (i,j,k). */
              flag = flag && 
                ((x[m]-x[i])*xn + (y[m]-y[i])*yn + (z[m]-z[i])*zn <= 0);
          if (flag) {
            printf("lower face indices: %d, %d, %d\n", i, j, k);
            F++;
          }
        }
  printf("A total of %d lower faces found.\n", F);
}
----------------------------------------------------------------------
\end{verbatim}
}%\small

You can compile this program at Levels 3 or Level 1.
At Level 3 accuracy, our program will correctly
detects all ${n \choose 3}$ triples;
at Level 1 accuracy, it is expected to miss many empty triples.
For example, when $n=5$, Level 3 gives all the 10 (= ${5 \choose 3}$) 
triangles, while Level 1 produces only 3.

%end
	% Example
% file: appendix_b.tex
% 	CORE Library
% 	$Id: appendix_c.tex,v 1.1.1.1 2007/04/09 23:35:54 exact Exp $ 

\newpage
\section{APPENDIX: Brief History}

\begin{description}
\item{Version 1.1}
(Dec 1998) The initial implementation
by Karamcheti, Li, Pechtchanski and Yap \cite{klpy:core:98}
was based on the \rexpr\ package,
designed by Dub\'e and Yap (circa 1993)
and rewritten by Ouchi \cite{ouchi:thesis}.  
Details about the underlying 
algorithms (especially in \BF) and their error analysis
may be found in the Ouchi's thesis \cite{ouchi:thesis}.

\item{Version 1.2}
(Sep 1999) Incorporates the BFMS root bound and other 
techniques to give significant speedup to the system. 

\item{Version 1.3}
(Sep 2000) Two improvements (new root bounds
and faster big number packages based on \lidia/\cln)
gave significant general speedup.  
More examples, including a hypergeometric function package
and a randomized geometric theorem prover.

\item{Version 1.4}
(Sep 2001) 
Introduced a floating point filter based on the BFS filter,
incremental square root computation,
improved precision-sensitive evaluation algorithms,
better numerical I/O support.
Our big integer and big rational packages
are now based on \gmp, away from \lidia/\cln.

\item{Version 1.5}
(Aug 2002) Improvements in speed from better root bounds (k-ary bounds),
CGAL compatibility changes, file I/O for large mathematical constants
(BigInt, BigFloat, BigRat), improved hypergeometric package.

\item{Version 1.6}
(June 2003)  The introduction of real algebraic numbers (a first
among such systems).
CORE is now issued under the the Q PUBLIC LICENSE (QPL),
concurrent with its being distributed with CGAL under commercial
licenses by Geometry Factory, the CGAL commercial spin-off.
Incorporated Polynomial and Sturm classes into Core Library.

\item{Version 1.7}
(Aug 2004) Introduced algebraic curves and bivariate polynomials.
An interactive version of Core Library called "InCore" is available.
Beginning basic graphic capability for display of curves.  
Restructuring of number classes (Expr, BigFloat, etc)
to have common reference counting and rep facilities.

\end{description}


	% Brief History

%%%%%%%%%%%%%%%%%%%%%%%%%%%%%%%%%%%%%%%%%%%%%%%%%%%%%%%%%%%%%%
\newpage
{\small
\bibliographystyle{abbrv}
\bibliography{tutorial}
}
\end{document}

